\documentclass[12pt, a4paper]{article}

\usepackage[top=2.3cm, bottom=2cm, left=2cm, right=2cm, headheight=15pt]{geometry}
\usepackage{graphicx}
\usepackage{adjustbox}
\usepackage{enumitem}
\usepackage[most]{tcolorbox}
%\usepackage{xcolor}
\usepackage[colorlinks=true]{hyperref}
\usepackage{fancyhdr}
\pagestyle{fancy}
\usepackage{caption}
\renewcommand{\contentsname}{Indice}
\renewcommand{\sectionmark}[1]{\markboth{#1}{#1}}
\graphicspath{{images/}} % COSE DA FARE: Directory "images" per le foto da inserire nel documento!



\title{
    \Huge Ingegneria del Software\\
    \vspace{.1cm}
    \Large Applicazione per la gestione di una biblioteca}
\author{
    \textbf{Gruppo 9}\\
    Luisa Genovese\\
    Erica Brancaccio\\
    Paolo Alfé\\
    Francesco Altieri}
\date{\today}

\begin{document}

    \fancyhf{}
	\fancyhead[R]{pagina \thepage}
    \maketitle
    \thispagestyle{empty}
    \newpage

    \pagestyle{fancy}
    \hypersetup{linkcolor=black}
    \tableofcontents
    \newpage

    \section*{Descrizione introduttiva}
        ...
    
    \section*{Mockup}
        ...
	\addcontentsline{toc}{section}{Mockup}
    \newpage

    \fancyhf{}
    \fancyhead[R]{page \thepage}
    \fancyhead[L]{\nouppercase{\leftmark}}
    \section{Ingegneria dei Requisiti - SRS}
    \subsection{Tabella di categorizzazione dei requisiti}
        ...

    \newpage
    \hypersetup{linkcolor=blue}
    \subsection{Requisiti Funzionali}
        \begin{tcolorbox}[breakable, colback=white,colframe=black!80!white,title=\textbf{Funzionalità individuali (IF)}]
            \begin{itemize}[itemsep=6pt]
                \hypertarget{IF-1}{\item[\textbf{IF-1}]}\textbf{Inserimento di un libro}
                \\...
                                      
                \item[\textbf{IF-2}]\textbf{Modifica di un libro}
                \\...

                \item[\textbf{IF-3}]\textbf{Cancellazione di un libro}
                \\...

                \item[\textbf{IF-4}]\textbf{Visualizzazione lista libri per titolo}
                \\...

                \item[\textbf{IF-5}]\textbf{Ricerca di un libro}
                \\...

                \item[\textbf{IF-6}]\textbf{Inserimento di un utente}
                \\...

                \item[\textbf{IF-7}]\textbf{Modifica di un utente}
                \\...

                \item[\textbf{IF-8}]\textbf{Cancellazione di un utente}
                \\...

                \item[\textbf{IF-9}]\textbf{Inserimento di un utente}
                \\...
            \end{itemize}
        \end{tcolorbox}
        \begin{tcolorbox}[breakable, colback=white,colframe=black!80!white,title=\textbf{Esigenze di Dati e Informazioni (DF)}]
            \begin{itemize}[itemsep=6pt]
                \item[\textbf{DF-1}]\textbf{...}
				\\
                    
                \item[\textbf{DF-2}]\textbf{...}
                \\vedi \hyperlink{IF-1}{IF-1}
            \end{itemize}
        \end{tcolorbox}
        \begin{tcolorbox}[breakable, colback=white,colframe=black!80!white,title=\textbf{Interfaccia Utente (UI)}]
            \begin{itemize}[itemsep=6pt]
                \item[\textbf{UI-1}]\textbf{...}                
				\\...
                    
                \item[\textbf{UI-2}]\textbf{...}
                \\...
            \end{itemize}
        \end{tcolorbox}

    \subsection{Requisiti Non Funzionali}
        L'applicazione non prevede requisiti di tipo non funzionale.

\end{document}
