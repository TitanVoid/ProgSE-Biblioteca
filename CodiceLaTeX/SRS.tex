\documentclass[12pt, a4paper]{article}

\usepackage[top=2.3cm, bottom=2cm, left=2cm, right=2cm, headheight=15pt]{geometry}
\usepackage{graphicx}
\usepackage{svg}
\usepackage[italian]{babel}
\usepackage{lastpage}
\usepackage{adjustbox}
\usepackage{enumitem}
\usepackage[most]{tcolorbox}
%\usepackage{xcolor}
\usepackage[colorlinks=true]{hyperref}
\usepackage{fancyhdr}
\pagestyle{fancy}
\usepackage{caption}
\renewcommand{\contentsname}{Indice}
\renewcommand{\sectionmark}[1]{\markboth{#1}{#1}}


\begin{document}
    \begin{titlepage}
        \fancyhf{}
		\centering
		
		{\Large \textsc{Ingegneria del Software - A.A. 2025/2026}}\\[.4cm]
        {\Large Applicazione per la gestione di una Biblioteca universitaria}\\[4cm]
		
		{\huge \textbf{Documento di Specifica dei\\Requisiti - SRS}}\\[0.5cm]
		
		\large { 
			\textit{
            Luisa Genovese\\
            Erica Brancaccio\\
            Paolo Alfé\\
            Francesco Altieri}
		}
		
		\vspace{13.5cm}
		
		{\large \today}
		
		
	\end{titlepage}
    
    \fancyhf{}
    \fancyhead[R]{pagina \thepage\ di \pageref*{LastPage}}
    \pagestyle{fancy}
    \hypersetup{linkcolor=black}
    \tableofcontents
    \newpage

    \section*{Descrizione introduttiva}
    \addcontentsline{toc}{section}{Descrizione introduttiva}
        \subsection*{Obiettivo}
            Si vuole realizzare un software per la gestione di una biblioteca universitaria.\vspace{.2cm}
            \\Questo vuole garantire all’utilizzatore del sistema, in modo semplice ed intuitivo, la totale gestione del catalogo di libri della biblioteca (mettendo a disposizione funzioni di aggiunta, cancellazione e modifica di un libro presente all’interno dell’archivio).
            \\Inoltre, mediante l'utilizzo di questo software è garantita anche la gestione di tutti gli utenti che hanno richiesto dei prestiti, tenendo traccia dei tempi di restituzione ed evidenziando eventuali ritardi.\vspace{.2cm}
            \\Si rende disponibile un’interfaccia grafica che permette all’utilizzatore di usufruire delle funzionalità offerte dal sistema.
        \subsection*{Stakeholder}
            Gli stakeholder principali del progetto si possono dividere in due categorie:
            \begin{itemize}[label=$\bullet$]
                \item \textbf{Stakeholder Interni:} coloro che interagiscono direttamente (o a stretto contatto) con l'applicazione per svolgere le loro funzioni, oppure che definiscono le regole che determinano il funzionamento stesso dell'applicazione; 
                \item \textbf{Stakeholder Esterni:} coloro che non utilizzano direttamente il software, ma che sono influenzati dalle sue funzionalità.
            \end{itemize}
            \subsubsection*{Stakeholder Interni}
                L'\textbf{Amministratore} della biblioteca ha l'interesse più alto nel progetto, poiché è colui che dovrà utilizzare l'applicazione per la gestione completa della biblioteca.\vspace{.2cm}
                \\Un altro stakeholder interno è il \textbf{Responsabile/Direttore} stesso della biblioteca che, pur non utilizzando l'applicazione, definisce le regole di business che ne determinano il funzionamento.
            \subsubsection*{Stakeholder Esterni}
                L'\textbf{Utente} (ovvero lo \textbf{Studente universitario}), sebbene non usi direttamente l'interfaccia grafica, è il beneficiario finale del sistema.

    \newpage
    \hypersetup{linkcolor=blue}
    \fancyhf{}
    \fancyhead[R]{pagina \thepage\ di \pageref*{LastPage}}
    \fancyhead[L]{\nouppercase{\leftmark}}
    \section{Ingegneria dei Requisiti - SRS}
    \subsection{Requisiti Funzionali}
        \begin{tcolorbox}[breakable, colback=white,colframe=black!80!white,title=\textbf{Funzionalità individuali - IF}]
            \begin{itemize}[itemsep=6pt]
                \hypertarget{IF-1}{\item[\textbf{IF-1}]}\textbf{Inserimento di un libro}
                \\L’applicazione consente all’amministratore di inserire un libro all’interno della lista dei libri.\vspace{.2cm}
                \\Premendo l’apposito pulsante, comparirà una finestra aggiuntiva in cui si dovranno inserire tutti i campi necessari, coerentemente a quanto definito in \hyperlink{DF-2}{DF-2}. 
                \\Al termine dell’inserimento, sarà possibile confermare oppure annullare l’operazione mediante i rispettivi pulsanti. 
                \\Nel caso in cui non siano stati inseriti tutti i campi necessari, oppure nel caso di inserimento di campi errati (vedere \hyperlink{DF-2}{DF-2} per le informazioni relative al formato dei dati), non sarà possibile confermare l’inserimento. 
                \\Se il codice identificativo univoco fornito in input dovesse corrispondere ad un libro già presente nella lista, l'operazione di inserimento sarà bloccata e l’amministratore verrà notificato dell’esistenza di un duplicato. 
                      
                \hypertarget{IF-2}{\item[\textbf{IF-2}]}\textbf{Modifica di un libro}
                \\L’applicazione consente all’amministratore di modificare le informazioni associate ad un libro.\vspace{.2cm}
                \\Selezionando un libro dalla lista dei libri e premendo l’apposito pulsante, sarà possibile apportare modifiche ad uno o più dei suoi campi, ad eccezione del campo “Codice Identificativo”.
                \\Sarà possibile salvare le modifiche effettuate oppure annullare l’operazione tramite appositi pulsanti.
                \\Il sistema dovrà impedire il salvataggio delle modifiche qualora uno o più dei campi (definiti in \hyperlink{DF-2}{DF-2}) dovessero essere lasciati vuoti. 

                \hypertarget{IF-3}{\item[\textbf{IF-3}]}\textbf{Cancellazione di un libro}
                \\L’applicazione consente all’amministratore di rimuovere un libro dalla lista dei libri.\vspace{.2cm}
                \\L'operazione viene avviata selezionando un libro presente nella lista dei libri e premendo l’apposito pulsante di cancellazione.
                \\Il sistema mostrerà un messaggio di conferma all’amministratore, offrendogli la possibilità di portare a termine o annullare l’operazione tramite i rispettivi pulsanti.
                \\La rimozione di un libro dalla lista dei libri non deve comportare la cancellazione dello storico dei prestiti precedentemente associati a quel volume, né annullare gli eventuali prestiti attivi al momento della rimozione.
                \\L'operazione, una volta confermata, è irreversibile. 

                \hypertarget{IF-4}{\item[\textbf{IF-4}]}\textbf{Visualizzazione lista libri per titolo}
                \\La visualizzazione dei libri contenuti nella lista dei libri avviene sempre in ordine alfabetico ascendente (A-Z), rispetto al loro titolo.\vspace{.2cm}
                \\L’ordinamento della lista dovrà essere mantenuto dopo ogni operazione di aggiunta, modifica o rimozione di un libro presente nella lista dei libri. 

                \hypertarget{IF-5}{\item[\textbf{IF-5}]}\textbf{Ricerca di un libro}
                \\L’applicazione consente all’amministratore di ricercare un libro all’interno della lista dei libri.\vspace{.2cm}
                \\L’operazione potrà avvenire digitando nella barra di ricerca uno tra i seguenti campi del libro da ricercare: 
                \begin{itemize}[label=$\bullet$]
                    \item Una sottostringa del titolo; 
                    \item Una sottostringa del nome dell’autore;
                    \item Una sottostringa del cognome dell'autore;
                    \item Una sottostringa del codice identificativo. 
                \end{itemize}
                La corrispondenza dovrà ignorare la distinzione tra maiuscole e minuscole.
                \\Qualora la ricerca non dovesse produrre alcuna corrispondenza, all'amministratore verrà presentata una tabella vuota.
                \\In caso di risultati multipli, i libri restituiti verranno mostrati secondo il criterio d’ordinamento definito in \hyperlink{IF-4}{IF-4}. 

                \hypertarget{IF-6}{\item[\textbf{IF-6}]}\textbf{Inserimento di un utente}
                \\L’applicazione consente all’amministratore di inserire un utente all’interno della lista dei libri.\vspace{.2cm}
                \\Premendo l’apposito pulsante, comparirà una finestra aggiuntiva in cui si dovranno inserire tutti i campi necessari, coerentemente a quanto definito in \hyperlink{DF-3}{DF-3}.
                \\Al termine dell’inserimento, sarà possibile confermare oppure annullare l’operazione mediante i rispettivi pulsanti.
                \\Nel caso in cui non siano stati inseriti tutti i campi necessari, oppure nel caso di inserimento di campi errati (vedere \hyperlink{DF-3}{DF-3} per le informazioni relative al formato dei dati), non sarà possibile confermare l’inserimento.
                \\Se la matricola fornita in input dovesse corrispondere ad un utente già presente nella lista, l'operazione di inserimento sarà bloccata e l’amministratore verrà notificato dell’esistenza di un duplicato. 
                
                \hypertarget{IF-7}{\item[\textbf{IF-7}]}\textbf{Modifica di un utente}
                \\L’applicazione consente all’amministratore di modificare le informazioni associate ad un utente.\vspace{.2cm}
                \\Selezionando un utente dalla lista degli utenti e premendo l’apposito pulsante, sarà possibile apportare modifiche ad uno o più dei suoi campi, ad eccezione del campo “Matricola”.
                \\Sarà possibile salvare le modifiche effettuate oppure annullare l’operazione tramite appositi pulsanti.
                \\Il sistema dovrà impedire il salvataggio delle modifiche qualora uno o più dei campi (definiti in \hyperlink{DF-3}{DF-3}) dovessero essere lasciati vuoti. 

                \hypertarget{IF-8}{\item[\textbf{IF-8}]}\textbf{Cancellazione di un utente}
                \\L’applicazione consente all’amministratore di rimuovere un utente dalla lista degli utenti.\vspace{.2cm}
                \\L'operazione viene avviata selezionando un utente presente nella lista degli utenti e premendo l’apposito pulsante di cancellazione.
                \\Il sistema mostrerà un messaggio di conferma all’amministratore, offrendogli la possibilità di portare a termine o annullare l’operazione tramite i rispettivi pulsanti.
                \\La rimozione di un utente dalla lista degli utenti non deve comportare la cancellazione dello storico dei prestiti precedentemente associati a quell'utente né annullare gli eventuali prestiti attivi al momento della rimozione.
                \\L'operazione, una volta confermata, è irreversibile. 

                \hypertarget{IF-9}{\item[\textbf{IF-9}]}\textbf{Visualizzazione degli utenti per cognome e nome}
                \\La visualizzazione degli utenti contenuti nella lista degli utenti avviene sempre in ordine alfabetico ascendente (A-Z), rispetto al loro cognome e, a parità di cognome, al loro nome.\vspace{.2cm}
                \\L’ordinamento della lista dovrà essere mantenuto dopo ogni operazione di aggiunta, modifica o rimozione di un utente presente nella lista degli utenti. 

                \hypertarget{IF-10}{\item[\textbf{IF-10}]}\textbf{Ricerca di un utente}
                \\L’applicazione consente all’amministratore di ricercare un utente all’interno della lista degli utenti.\vspace{.2cm}
                \\L’operazione potrà avvenire digitando nella barra di ricerca uno tra i seguenti campi dell’utente da ricercare: 
                \begin{itemize}[label=$\bullet$]
                    \item Una sottostringa del cognome; 
                    \item Una sottostringa della matricola. 
                \end{itemize}
                La corrispondenza dovrà ignorare la distinzione tra maiuscole e minuscole.
                \\Qualora la ricerca non dovesse produrre alcuna corrispondenza, all'amministratore verrà presentata una tabella vuota.
                \\In caso di risultati multipli, gli utenti restituiti verranno mostrati secondo il criterio d’ordinamento definito in \hyperlink{IF-9}{IF-9}. 

                \hypertarget{IF-11}{\item[\textbf{IF-11}]}\textbf{Registrazione di un prestito}
                \\L’applicazione consente all’amministratore di registrare un nuovo prestito.\vspace{.2cm}
                \\Premendo l’apposito pulsante, comparirà una finestra aggiuntiva in cui si dovranno inserire tutti i campi necessari, coerentemente a quanto definito in \hyperlink{DF-4}{DF-4}.
                \\Al termine dell’inserimento, sarà possibile confermare oppure annullare l’operazione mediante i rispettivi pulsanti.
                \\All’atto di una richiesta di prestito, il sistema dovrà svolgere automaticamente le seguenti operazioni:
                \begin{itemize}[label=$\bullet$]
                    \item Aggiornamento del numero di copie disponibili del libro prestato; 
                    \item Aggiornamento della lista dei libri associati all’utente, aggiungendo il volume appena preso in prestito; 
                    \item Aggiunta del prestito dalla schermata di visualizzazione dei prestiti attivi.
                \end{itemize}
                Nel caso in cui non siano stati inseriti tutti i campi necessari, oppure nel caso di inserimento di campi errati (vedere \hyperlink{DF-4}{DF-4} per le informazioni relative al formato dei dati), non sarà possibile confermare la registrazione.
                \\Se il codice identificativo del libro e/o la matricola dell’utente forniti in input non dovessero essere presenti nelle rispettive liste, l'operazione di registrazione sarà bloccata. 

                \hypertarget{IF-12}{\item[\textbf{IF-12}]}\textbf{Visualizzazione dei prestiti attivi per data prevista di restituzione}
                \\La visualizzazione dei prestiti attivi contenuti nella lista dei prestiti avviene sempre in ordine cronologico crescente, rispetto alla data di restituzione.\vspace{.2cm}
                \\L’ordinamento della lista dovrà essere mantenuto dopo ogni operazione di registrazione, di estensione di un prestito o di restituzione di un libro. 

                \hypertarget{IF-13}{\item[\textbf{IF-13}]}\textbf{Registrazione di una restituzione}
                \\L’applicazione permette all’amministratore di registrare la restituzione di un libro precedentemente preso in prestito.\vspace{.2cm}
                \\L’amministratore dovrà selezionare, dalla lista dei prestiti attivi, il prestito da chiudere e avviare l’operazione di restituzione premendo l’apposito pulsante.
                \\All’atto di una restituzione, il sistema dovrà svolgere automaticamente le seguenti operazioni: 
                \begin{itemize}[label=$\bullet$]
                    \item Aggiornamento del numero di copie disponibili del libro restituito;
                    \item Aggiornamento della lista dei libri associati all’utente, rimuovendo il volume appena restituito;
                    \item Rimozione del prestito dalla schermata di visualizzazione dei prestiti attivi.
                \end{itemize}
                Tutti i prestiti conclusi saranno comunque archiviati all’interno di un file, coerentemente a quanto definito in \hyperlink{DF-1}{DF-1}. 

                \hypertarget{IF-14}{\item[\textbf{IF-14}]}\textbf{Filtraggio dei prestiti}
                \\L’applicazione supporta, nel caso della lista dei libri, una funzionalità di ricerca avanzata mediante criteri di filtro multipli e combinati.\vspace{.2cm}
                \\Tali criteri di filtro potranno essere applicati autonomamente, come alternativa alla ricerca testuale tradizionale (definita in \hyperlink{IF-5}{IF-5}).\\Dovrà essere supportare la possibilità di filtrare la lista dei libri sui seguenti campi: 
                \begin{itemize}[label=$\bullet$]
                    \item Anno di pubblicazione: sarà possibile visualizzare tutti i libri pubblicati a partire da/prima di un determinato anno, oppure in un intervallo;
                    \item Disponibilità: potranno essere visualizzati tutti i libri il cui stato risulta essere “Disponibile” oppure “Non Disponibile”;
                    \item Numero di copie disponibili: sarà possibile visualizzare i libri che hanno una disponibilità maggiore/minore di un determinato numero di copie, oppure compresa in un intervallo.
                \end{itemize}
                I risultati della ricerca dovranno essere presentati rispettando l'ordinamento definito in \hyperlink{IF-4}{IF-4}. 

                \hypertarget{IF-15}{\item[\textbf{IF-15}]}\textbf{Estensione prestito}
                \\L’applicazione consente di estendere la data di restituzione prevista per un prestito attivo.\vspace{.2cm}
                \\In caso di esplicita richiesta da parte dell’utente, l’amministratore dovrà poter selezionare il prestito di interesse e modificarne il campo relativo alla data di restituzione (rispettandone il formato, definito in \hyperlink{DF-4}{DF-4}).
                \\L’estensione sarà soggetta a delle limitazioni temporali e numeriche: 
                \begin{itemize}[label=$\bullet$]
                    \item Il servizio potrà essere richiesto dagli utenti soltanto nel caso in cui non sia stata superata la data di restituzione prevista;
                    \item Per ogni prestito è ammessa un’unica richiesta di estensione.
                \end{itemize}
            \end{itemize}
        \end{tcolorbox}

        \begin{tcolorbox}[breakable, colback=white,colframe=black!80!white,title=\textbf{Business Flow - BF}]
            \begin{itemize}[itemsep=6pt]
                \hypertarget{BF-1}{\item[\textbf{BF-1}]}\textbf{Segnalazione ritardi}
                \\L’applicazione permette di segnalare all'utente i prestiti che superano la data prevista di restituzione, attraverso la loro identificazione automatica e l'intervento manuale dell’amministratore.\vspace{.2cm}
                \\Un prestito è considerato “in ritardo” se la data corrente è successiva alla data prevista di restituzione.
                \\Ogni prestito per il quale avviene ciò viene segnalato in maniera visiva all’amministratore (coerentemente a quanto definito in \hyperlink{UI-4}{UI-4}).
                \\L’amministratore, attraverso l'informazione relativa al numero di telefono che viene memorizzata dal sistema (vedere \hyperlink{DF-3.1}{DF-3.1}), una volta presa visione del ritardo chiamerà l’utente che ha effettuato il prestito, notificandolo della mancata restituzione del libro.
                \\Nel caso in cui la chiamata sia stata effettuata con successo, l’amministratore annoterà l’avvenuta segnalazione sull’applicazione, tramite la selezione di un’apposita casella.   
                
                \hypertarget{BF-2}{\item[\textbf{BF-2}]}\textbf{Prestito di un libro}
                \\La biblioteca offre all’utente la possibilità di ottenere in prestito un libro presente nel catalogo.
                \\Tramite l’applicazione, l’amministratore potrà successivamente procedere con la registrazione del prestito solo se vengono rispettate le seguenti condizioni:
                \begin{itemize}[label=$\bullet$]
                    \item L’utente deve essere presente nell’elenco di utenti della biblioteca;
                    \item L’utente non deve avere più di tre prestiti attivi contemporaneamente;
                    \item Il libro richiesto deve avere un numero di copie disponibili maggiore di zero. 
                \end{itemize}
                \vspace{.2cm}All’atto della richiesta di prestito da parte dell’utente, l’amministratore dovrà dapprima verificare la sua presenza all’interno dell’elenco di utenti gestito dal sistema.
                \\Qualora l’utente non dovesse essere presente in tale elenco, quest’ultimo fornirà i propri dati personali all’amministratore, che lo inserirà mediante l’operazione definita in \hyperlink{IF-6}{IF-6}.
                \\Una volta ottenute (o in alternativa, verificate) le informazioni personali dell’utente, esso mostrerà all’amministratore il libro o i libri che intende prendere in prestito.
                \\Per ogni libro richiesto, l’amministratore dovrà: 
                \begin{itemize}[label=$\bullet$]
                    \item Verificare che esso sia presente nel catalogo mediante una ricerca (vedere \hyperlink{IF-5}{IF-5});
                    \item Verificare che il libro sia disponibile;
                    \item Verificare che l’utente non superi il limite massimo di prestiti.
                \end{itemize}
                Se tutte le condizioni sono soddisfatte, l'amministratore, tramite l'applicazione, registra l'avvenuto prestito all'Utente (come definito in \hyperlink{IF-11}{IF-11}).
                \\Se uno dei libri richiesti in prestito non è disponibile, l'Amministratore notificherà l'indisponibilità all'utente.
                \\Se l'utente ha già tre prestiti attivi, l’amministratore notificherà all'utente la necessità di restituire i libri in suo possesso prima di poterne prendere altri in prestito. 
                
                \hypertarget{BF-3}{\item[\textbf{BF-3}]}\textbf{Restituzione di un libro}
                \\La biblioteca offre all’utente la possibilità di restituire un libro preso precedentemente in prestito.\vspace{.2cm}
                \\Tramite l’applicazione, l’amministratore potrà successivamente procedere con la registrazione dell'avvenuta restituzione.
                \\All’atto della richiesta di restituzione, l’utente dovrà identificarsi all’amministratore e presentare il libro o i libri che intende restituire.
                \\Una volta ottenuti i dati necessari, l’amministratore verificherà l’esistenza di un prestito attivo che corrisponde alle informazioni fornite dall’utente, accedendo alla schermata di visualizzazione dei prestiti attivi (vedere \hyperlink{UI-4}{UI-4}).
                \\Una volta identificato il prestito corrispondente, l’amministratore lo selezionerà e procederà con la registrazione dell’avvenuta restituzione (definita in \hyperlink{IF-13}{IF-13}). 
            \end{itemize}
        \end{tcolorbox}
        
        \begin{tcolorbox}[breakable, colback=white,colframe=black!80!white,title=\textbf{Esigenze di Dati e Informazioni - DF}]
            \begin{itemize}[itemsep=6pt]
                \hypertarget{DF-1}{\item[\textbf{DF-1}]}\textbf{Archiviazione su file}
                \\L’intero archivio della biblioteca dovrà essere memorizzato, in maniera automatica, all’interno di una serie di file.\vspace{.2cm}
                \\I dati saranno suddivisi e memorizzati in tre file distinti: 
                \begin{itemize}[label=$\bullet$]
                    \item Un file per la memorizzazione dell’elenco dei libri;
                    \item Un file per la memorizzazione dell’elenco degli utenti;
                    \item Un file per la memorizzazione dell’elenco dei prestiti. 
                \end{itemize}
                L’archiviazione dovrà avvenire ogni qualvolta il software viene chiuso.
                \\Al riavvio dell’applicazione, il sistema dovrà eseguire automaticamente la lettura e il caricamento in memoria del contenuto di tutti e tre i file.      
                
                \hypertarget{DF-2}{\item[\textbf{DF-2}]}\textbf{Informazioni di un libro}
                \\Il sistema deve memorizzare, nel formato appropriato, tutte le informazioni relative ai libri presenti nella lista dei libri della biblioteca.\vspace{.2cm}
                \\I dati che dovranno essere memorizzati sono: 
                \begin{itemize}[label=$\bullet$]
                    \item Titolo;
                    \item Lista degli autori (nome e cognome); 
                    \item Anno di pubblicazione; 
                    \item Codice identificativo (univoco); 
                    \item Numero di copie disponibili. 
                \end{itemize}
                Nello specifico: 
                \begin{itemize}[label=$\bullet$]
                    \item Il campo “Titolo” dovrà essere una stringa, con una lunghezza massima di cento caratteri;
                    \item I campi “Nome” e “Cognome” di ogni autore dovranno essere stringhe di caratteri alfabetici (con una lunghezza massima di venticinque caratteri);
                    \item Il campo “Anno di pubblicazione” dovrà essere un numero a quattro cifre, compreso tra zero e l’anno corrente;
                    \item Il campo “Codice identificativo” dovrà essere il codice ISBN del libro, e dunque conforme allo standard ISO 2108: 2017;
                    \item Il campo “Numero di copie disponibili” dovrà essere un numero compreso tra zero e cento (definita come disponibilità massima per un singolo volume). 
                \end{itemize}
                
                \hypertarget{DF-3}{\item[\textbf{DF-3}]}\textbf{Informazioni dell'utente}
                \\Il sistema deve memorizzare, nel formato appropriato, tutte le informazioni relative agli utenti della biblioteca.\vspace{.2cm}
                I dati che dovranno essere memorizzati sono: 
                \begin{itemize}[label=$\bullet$]
                    \item Nome;
                    \item Cognome;
                    \item Matricola (univoca);
                    \item E-mail istituzionale; 
                    \item Lista dei libri attualmente in prestito e data prevista per la restituzione (di ciascun libro).
                \end{itemize}
                Nello specifico: 
                \begin{itemize}[label=$\bullet$]
                    \item I campi “Nome” e “Cognome” dovranno essere stringhe di caratteri alfabetici (con una lunghezza massima di venticinque caratteri);
                    \item Il campo “Matricola” dovrà essere una stringa di dieci cifre numeriche;
                    \item Il campo “E-mail” dovrà aderire allo standard RFC 5322, e come campo “dominio” dovrà avere “studenti.unisa.it”. 
                \end{itemize}

                \hypertarget{DF-4}{\item[\textbf{DF-4}]}\textbf{Informazioni sui prestiti}
                \\Il sistema deve memorizzare, nel formato appropriato, tutte le informazioni relative ai prestiti effettuati agli utenti della biblioteca.\vspace{.2cm}
                \\I dati che dovranno essere memorizzati sono: 
                \begin{itemize}[label=$\bullet$]
                    \item Identificativo (univoco) dell’utente che ha richiesto il prestito;
                    \item Identificativo (univoco) del libro prestato; 
                    \item Data di avvenuto (inizio) prestito; 
                    \item Data prevista per la restituzione;
                    \item Data effettiva di restituzione (solo per i prestiti completati). 
                \end{itemize}
                Nello specifico: 
                \begin{itemize}[label=$\bullet$]
                    \item L’identificativo dell’utente corrisponderà alla propria matricola (vedere \hyperlink{DF-2}{DF-2} per le specifiche sul formato); 
                    \item L’identificativo del libro corrisponderà al suo codice ISBN (vedere \hyperlink{DF-3}{DF-3} per le specifiche sul formato);
                    \item I campi relativi alle date dovranno avere il seguente formato: “AAAA-MM-GG”. 
                \end{itemize}
            \end{itemize}
        \end{tcolorbox}
        
        \begin{tcolorbox}[breakable, colback=white,colframe=black!80!white,title=\textbf{Interfaccia Utente - UI}]
            \begin{itemize}[itemsep=6pt]
                \hypertarget{UI-1}{\item[\textbf{UI-1}]}\textbf{Interfaccia di tipo grafico}
                \\L’amministratore potrà interagire con l’applicazione mediante un’interfaccia di tipo grafico, evitando l’uso diretto di interfacce testuali o della linea di comando.\vspace{.2cm}
                \\L'uso di un'interfaccia consente, mediante un insieme di schermate di visualizzazione sequenziali (vedere \hyperlink{UI-2}{UI-2}, \hyperlink{UI-3}{UI-3} e \hyperlink{UI-4}{UI-4} per la loro descrizione dettagliata), la navigazione tra le varie funzionalità messe a disposizione dal sistema. 
                
                \hypertarget{UI-2}{\item[\textbf{UI-2}]}\textbf{Visualizzazione lista dei libri}
                \\L'applicazione dovrà includere una schermata dedicata alla visualizzazione e alla gestione dell’elenco completo dei libri presenti nella lista dei libri.
                \\Di default, all’apertura dell’applicazione verrà mostrata questa schermata.\vspace{.2cm}
                \\L'elenco dei libri verrà presentato all’amministratore in formato tabellare, rispettando l’ordinamento definito in \hyperlink{IF-4}{IF-4}.
                \\La schermata supporterà la paginazione, limitando la visualizzazione ad un massimo di venti libri per pagina.
                \\Saranno previste le seguenti funzionalità di interazione: 
                \begin{enumerate}
                    \item Barra di ricerca: sarà disponibile un campo di testo dedicato per effettuare la ricerca testuale (descritta in \hyperlink{IF-5}{IF-5}); 
                    \item Filtri avanzati (vedere \hyperlink{IF-14}{IF-14}): l’amministratore avrà modo di aprire, premendo il pulsante “Filtri”, una schermata apposita dove sarà possibile selezionare uno o più criteri, in base ai quali verrà limitata la visualizzazione dell’elenco;
                    \item Pulsanti operativi per l’esecuzione delle operazioni principali:
                    \begin{itemize}[label=$\bullet$]
                        \item Pulsante per accedere alla visualizzazione delle informazioni complete relative ai singoli libri (vedere \hyperlink{DF-2}{DF-2});
                        \item Pulsante “Aggiungi” per l’inserimento di un nuovo libro all’interno dell’elenco (vedere \hyperlink{IF-1}{IF-1}); 
                        \item Pulsante “Modifica” per la modifica delle informazioni relative al libro selezionato (vedere \hyperlink{IF-2}{IF-2});
                        \item Pulsante “Rimuovi” per la rimozione di un libro dall’elenco (vedere \hyperlink{IF-3}{IF-3}). 
                    \end{itemize} 
                \end{enumerate}
                Tutte le operazioni di rimozione dovranno essere precedute da una schermata di conferma, che richiede l’esplicito consenso dall’amministratore (ottenuto mediante la pressione del pulsante “Sì”) prima dell’esecuzione.
                \\Inoltre, tutte le operazioni di gestione della lista dei libri, nonché le stesse schermate di conferma, dovranno includere un pulsante di annullamento che permette di tornare alla schermata precedente. 
                
                \hypertarget{UI-3}{\item[\textbf{UI-3}]}\textbf{Visualizzazione lista degli utenti}
                \\L'applicazione dovrà includere una schermata dedicata alla visualizzazione e alla gestione dell’elenco completo degli utenti presenti nella lista degli utenti.\vspace{.2cm}
                \\L'elenco degli utenti verrà presentato all’amministratore in formato tabellare, rispettando l’ordinamento definito in \hyperlink{IF-9}{IF-9}.
                \\La schermata supporterà la paginazione, limitando la visualizzazione ad un massimo di venti utenti per pagina.
                \\Saranno previste le seguenti funzionalità di interazione: 
                \begin{enumerate}
                    \item Barra di ricerca: sarà disponibile un campo di testo dedicato per effettuare la ricerca testuale (descritta in \hyperlink{IF-10}{IF-10});
                    \item Pulsanti operativi per l’esecuzione delle operazioni principali:
                    \begin{itemize}[label=$\bullet$]
                        \item Pulsante per accedere alla visualizzazione delle informazioni complete relative ai singoli utenti (vedere \hyperlink{DF-3}{DF-3}); 
                        \item Pulsante “Aggiungi” per l’inserimento di un nuovo utente all’interno dell’elenco (vedere \hyperlink{IF-6}{IF-6});
                        \item Pulsante “Modifica” per la modifica delle informazioni relative all’utente selezionato (vedere \hyperlink{IF-7}{IF-7}); 
                        \item Pulsante “Rimuovi” per la rimozione di un utente dall’elenco (vedere \hyperlink{IF-8}{IF-8}).
                    \end{itemize}
                \end{enumerate}
                Tutte le operazioni di rimozione dovranno essere precedute da una schermata di conferma, che richiede l’esplicito consenso dall’amministratore (ottenuto mediante la pressione del pulsante “Sì”) prima dell’esecuzione.
                \\Inoltre, tutte le operazioni di gestione della lista degli utenti, nonché le stesse schermate di conferma, dovranno includere un pulsante di annullamento che permette di tornare alla schermata precedente.  
                
                \hypertarget{UI-4}{\item[\textbf{UI-4}]}\textbf{Visualizzazione lista dei prestiti}
                \\L'applicazione dovrà includere una schermata dedicata alla visualizzazione e alla gestione dell’elenco completo dei prestiti.\vspace{.2cm}
                \\L'elenco dei prestiti verrà presentato all’utente in formato tabellare, rispettando l’ordinamento definito in \hyperlink{IF-12}{IF-12}.
                \\La schermata supporterà la paginazione, limitando la visualizzazione ad un massimo di venti prestiti per pagina.
                \\Saranno previste le seguenti funzionalità di interazione: 
                \begin{enumerate}
                    \item Messa in evidenza dei ritardi: nel caso dei prestiti che presentano un ritardo, la data prevista per la restituzione verrà evidenziata in rosso, in modo tale da segnalare visivamente il ritardo all’amministratore.
                    \\Ad ogni prestito in ritardo sarà inoltre associata una casella, da spuntare in caso di avvenuta telefonata all’utente da parte dell’amministratore (vedere \hyperlink{BF-1}{BF-1}).
                    \item Pulsanti operativi per l’esecuzione delle operazioni principali:
                    \begin{itemize}[label=$\bullet$]
                        \item Pulsante “Prestiti” per la visualizzazione delle informazioni complete relative ai singoli prestiti (vedere \hyperlink{DF-4}{DF-4}); 
                        \item Pulsante “Inserisci” per la registrazione di un nuovo prestito (vedere \hyperlink{IF-11}{IF-11});
                        \item Pulsante “Estendi” per l’estensione di un prestito già esistente (vedere \hyperlink{IF-15}{IF-15}); 
                        \item Pulsante “Restituzione” per la registrazione di una restituzione (vedere \hyperlink{IF-13}{IF-13}). 
                    \end{itemize}
                \end{enumerate}
                Tutte le operazioni di registrazione dovranno essere precedute da una schermata di conferma, che richiede l’esplicito consenso dall’amministratore (ottenuto mediante la pressione del pulsante “Sì”) prima dell’esecuzione.
                \\Inoltre, tutte le operazioni di gestione della lista dei prestiti, nonché le stesse schermate di conferma, dovranno includere un pulsante di annullamento che permette di tornare alla schermata precedente.  
            \end{itemize}
        \end{tcolorbox}

    \newpage
    \subsection{Requisiti Non Funzionali}
        \begin{tcolorbox}[breakable, colback=white,colframe=black!80!white,title=\textbf{Requisiti Non Funzionali - FC}]
            \begin{itemize}[itemsep=6pt]
                \item[\textbf{FC-1}]\textbf{Affidabilità dei dati}
                \\Lo stato corrente dell'archivio (vedere \hyperlink{DF-2}{DF-2}, \hyperlink{DF-3}{DF-3} e \hyperlink{DF-4}{DF-4} per le specifiche sui dati memorizzati) dovrà essere salvato immediatamente prima della chiusura dell’applicazione, in modo che i dati possano essere protetti da perdite, corruzioni e modifiche non autorizzate.
                \\I dati dovranno essere sempre recuperabili. 
                
                \item[\textbf{FC-2}]\textbf{Intuitività e semplicità dell'interfaccia}
                \\L’interfaccia \hyperlink{UI-1}{UI-1} dovrà essere progettata col fine di massimizzare l’intuitività e la semplicità operativa.\vspace{.2cm}
                \\I due criteri utilizzati per la misurazione di tali aspetti sono: 
                \begin{enumerate}
                    \item Efficienza nell’eseguibilità: l’esecuzione delle operazioni principali sarà efficiente, in modo tale che il completamento di queste ultime non richieda più di quattro pressioni di pulsanti;
                    \item Coerenza del layout: Il layout dovrà essere chiaro, privo di eccessivo affollamento e ricco di elementi autoesplicativi, al fine di rendere la navigazione prevedibile e facilmente apprendibili. 
                \end{enumerate}
                Inoltre, i pulsanti di navigazione principali e i pulsanti di azione manterranno un posizionamento semanticamente coerente rispetto alla loro funzionalità, attraverso tutte le schermate operative.  
                 
            \end{itemize}
        \end{tcolorbox}

    \newpage
    \subsection{Tabella di categorizzazione dei requisiti}
        \begin{adjustbox}{width=.8\paperwidth, height=.8\paperheight, center, margin=0cm 0cm 0cm .5cm}
            \includegraphics{images/requisiti1.png}
        \end{adjustbox}
        
        \newpage
        \begin{figure}[h!]
            \begin{adjustbox}{width=.8\paperwidth, height=.8\paperheight, center, margin=0cm 0cm 0cm 1cm}
                \includegraphics{images/requisiti2.png}
            \end{adjustbox}
        \end{figure}

        \newpage
        \begin{figure}[h!]
            \begin{adjustbox}{width=.8\paperwidth, height=.25\paperheight, margin=0cm 0cm 0cm .5cm}
                \includegraphics{images/requisiti3.png}
            \end{adjustbox}
            \caption{Tabella di categorizzazione dei requisiti}
        \end{figure}
    
    \newpage
    \subsection{Diagramma UML dei Casi d'Uso}
        \begin{adjustbox}{width=.86\paperwidth, height=.8\paperheight, center, margin= 0cm 0cm 0cm .1cm}
            \includesvg{usecases.svg}    
        \end{adjustbox}
        \captionof{figure}{Diagramma UML dei Casi d'Uso}  
        
    \newpage
    \subsection{Descrizione dei Casi d'Uso}
    
        \begin{tcolorbox}[breakable, colback=white,colframe=black!80!white,title=\textbf{C1 - Visualizza lista libri}]
            \textbf{Attori presenti:} Amministratore.\vspace{.2cm}     
                
            \textbf{Precondizioni:} L’amministratore ha avviato l’applicazione;\\L’amministratore si trova su una schermata diversa da quella di visualizzazione dei libri;\\L'archivio è stato caricato correttamente dal file all'avvio del sistema.\vspace{.2cm}
                
            \textbf{Postcondizioni:} L’elenco dei libri viene visualizzato in ordine alfabetico, per titolo.\vspace{.2cm}

            \textbf{Flusso di eventi normale:}\vspace{-0.2cm}
            \begin{enumerate}
                \setlength\itemsep{-0.3em}
                \item L’amministratore preme il pulsante apposito per aprire il menù a tendina; 
                \item L’amministratore preme il pulsante “Libri”; 
                \item Viene visualizzato l’elenco dei libri; 
            \end{enumerate}

            \textbf{Flussi di eventi alternativi:}\vspace{-0.2cm}
            \begin{enumerate}
                \item[1a.] L'amministratore visualizza la schermata principale dell'applicazione;\vspace{-0.2cm}
                \begin{enumerate}
                    \item[1a.1] L'esecuzione riprende dal passo 2.
                \end{enumerate}
                \item[3a.] Viene visualizzata una tabella vuota.
            \end{enumerate}                
        \end{tcolorbox}

        \begin{tcolorbox}[breakable, colback=white,colframe=black!80!white,title=\textbf{C2 - Visualizza lista utenti}]
            \textbf{Attori presenti:} Amministratore.\vspace{.2cm}     
                
            \textbf{Precondizioni:} L’amministratore ha avviato l’applicazione;\\L’amministratore si trova su una schermata diversa da quella di visualizzazione degli utenti;\\L'archivio è stato caricato correttamente dal file all'avvio del sistema.\vspace{.2cm}
                
            \textbf{Postcondizioni:} L’elenco degli utenti viene visualizzato in ordine alfabetico, per cognome e (a parità di cognome) per nome.\vspace{.2cm}

            \textbf{Flusso di eventi normale:}\vspace{-0.2cm}
            \begin{enumerate}
                \setlength\itemsep{-0.3em}
                \item L’amministratore preme il pulsante apposito per aprire il menù a tendina; 
                \item L’amministratore preme il pulsante “Utenti”; 
                \item Viene visualizzato l’elenco degli utenti; 
            \end{enumerate}

            \textbf{Flussi di eventi alternativi:}\vspace{-0.2cm}
            \begin{enumerate}
                \item[1a.] L'amministratore visualizza la schermata principale dell'applicazione;\vspace{-0.2cm}
                \begin{enumerate}
                    \item[1a.1] L'esecuzione riprende dal passo 2.
                \end{enumerate}
                \item[3a.] Viene visualizzata una tabella vuota.
            \end{enumerate}       
        \end{tcolorbox}

        \begin{tcolorbox}[breakable, colback=white,colframe=black!80!white,title=\textbf{C3 - Visualizza lista prestiti}]
            \textbf{Attori presenti:} Amministratore.\vspace{.2cm}     
                
            \textbf{Precondizioni:} L’amministratore ha avviato l’applicazione;\\L’amministratore si trova su una schermata diversa da quella di visualizzazione dei prestiti;\\L'archivio è stato caricato correttamente dal file all'avvio del sistema. \vspace{.2cm}
                
            \textbf{Postcondizioni:} L’elenco dei prestiti viene visualizzato in ordine cronologico, per data prevista di restituzione.\vspace{.2cm}

            \textbf{Flusso di eventi normale:}\vspace{-0.2cm}
            \begin{enumerate}
                \setlength\itemsep{-0.3em}
                \item L’amministratore preme il pulsante apposito per aprire il menù a tendina; 
                \item L’amministratore preme il pulsante “Prestiti”; 
                \item Viene visualizzato l’elenco dei prestiti; 
            \end{enumerate}

            \textbf{Flussi di eventi alternativi:}\vspace{-0.2cm}
            \begin{enumerate}
                \item[1a.] L'amministratore visualizza la schermata principale dell'applicazione;\vspace{-0.2cm}
                \begin{enumerate}
                    \item[1a.1] L'esecuzione riprende dal passo 2.
                \end{enumerate}
                \item[3a.] Viene visualizzata una tabella vuota.
            \end{enumerate}       
        \end{tcolorbox}

        \begin{tcolorbox}[breakable, colback=white,colframe=black!80!white,title=\textbf{C4 - Inserisci libro}]
            \textbf{Attori presenti:} Amministratore.\vspace{.2cm}     
                
            \textbf{Precondizioni:} L’amministratore si trova sulla schermata di visualizzazione e gestione dei libri.\vspace{.2cm}
                
            \textbf{Postcondizioni:} Il libro viene aggiunto con successo alla lista dei libri.\vspace{.2cm}

            \textbf{Flusso di eventi normale:}\vspace{-0.2cm}
            \begin{enumerate}
                \setlength\itemsep{-0.3em}
                \item L’amministratore preme il pulsante “Aggiungi”;
                \item Viene mostrata una nuova finestra con i campi richiesti per l'inserimento (specificati in \hyperlink{DF-2}{DF-2});
                \item L’amministratore inserisce tutte le informazioni necessarie per l’inserimento del libro;
                \item L’amministratore preme il pulsante “Conferma”;
                \item Il sistema convalida i dati (formato corretto e campi obbligatori compilati);
                \item Il sistema verifica l’univocità dell’ISBN;
                \item Il sistema inserisce il nuovo libro all’interno dell’archivio;
                \item La finestra di inserimento dei campi viene chiusa;
                \item L’amministratore visualizza la lista dei libri aggiornata. 
            \end{enumerate}

            \textbf{Flussi di eventi alternativi:}\vspace{-0.2cm}
            \begin{enumerate}
                \setlength\itemsep{-0.3em}
                \item[4a.] L’amministratore preme il pulsante “Annulla”;\vspace{-0.2cm}
                \begin{enumerate}
                    \item[4a.1] L’esecuzione riprende dal passo 1.
                \end{enumerate}
                \item[5a.] Il sistema riscontra degli errori nel formato dei campi inseriti;\vspace{-0.2cm}
                \begin{enumerate}
                    \setlength\itemsep{-0.3em}
                    \item[5a.1] Viene visualizzato un messaggio di errore e vengono evidenziati i campi da correggere;
                    \item[5a.2] L’esecuzione riprende dal passo 3.
                \end{enumerate}
                \item[5b.] Il sistema si accorge che non sono stati inseriti tutti i campi obbligatori;\vspace{-0.2cm}
                \begin{enumerate}
                    \setlength\itemsep{-0.3em}
                    \item[5b.1] Viene visualizzato un messaggio di errore e vengono evidenziati i campi da inserire;
                    \item[5b.2] L’esecuzione riprende dal passo 3.
                \end{enumerate}
                \item[6a.] L’ISBN inserito corrisponde ad un libro già presente nell’archivio;\vspace{-0.2cm}
                \begin{enumerate}
                    \setlength\itemsep{-0.3em}
                    \item[6a.1] Viene visualizzato un messaggio di errore;
                    \item[6a.2] L’esecuzione riprende dal passo 3.
                \end{enumerate}
            \end{enumerate}       
        \end{tcolorbox}

        \begin{tcolorbox}[breakable, colback=white,colframe=black!80!white,title=\textbf{C5 - Modifica libro}]
            \textbf{Attori presenti:} Amministratore.\vspace{.2cm}     
                
            \textbf{Precondizioni:} L’amministratore si trova sulla schermata di visualizzazione e gestione dei libri;\\È presente almeno un libro nella lista dei libri;\\L'archivio è stato caricato correttamente dal file all'avvio del sistema. \vspace{.2cm}
                
            \textbf{Postcondizioni:} Uno o più campi del libro selezionato vengono modificati con successo.\vspace{.2cm}

            \textbf{Flusso di eventi normale:}\vspace{-0.2cm}
            \begin{enumerate}
                \setlength\itemsep{-0.3em}
                \item L’amministratore seleziona un libro presente nella lista dei libri;
                \item L’amministratore preme il pulsante “Modifica”;
                \item Viene mostrata una nuova finestra contenente i campi correnti del libro selezionato;
                \item L’amministratore modifica uno o più campi;
                \item L’amministratore preme il pulsante “Conferma”.
                \item Il sistema convalida i dati (formato corretto e campi obbligatori compilati);
                \item Il sistema aggiorna i dati del libro nell’archivio;
                \item La finestra di modifica dei campi viene chiusa;
                \item L’amministratore visualizza la lista dei libri aggiornata. 
            \end{enumerate}

            \textbf{Flussi di eventi alternativi:}\vspace{-0.2cm}
            \begin{enumerate}
                \setlength\itemsep{-0.3em}
                \item[5a.] L’amministratore preme il pulsante “Annulla”;\vspace{-0.2cm}
                \begin{enumerate}
                    \item[5a.1] L’esecuzione riprende dal passo 1.
                \end{enumerate}
                \item[6a.] Il sistema riscontra degli errori nel formato dei campi inseriti;\vspace{-0.2cm}
                \begin{enumerate}
                    \setlength\itemsep{-0.3em}
                    \item[6a.1] Viene visualizzato un messaggio di errore e vengono evidenziati i campi da correggere;
                    \item[6a.2] L’esecuzione riprende dal passo 4.
                \end{enumerate}
                \item[6b.] Il sistema si accorge che non sono stati inseriti tutti i campi obbligatori;\vspace{-0.2cm}
                \begin{enumerate}
                    \setlength\itemsep{-0.3em}
                    \item[6b.1] Viene visualizzato un messaggio di errore e vengono evidenziati i campi da inserire;
                    \item[6b.2] L’esecuzione riprende dal passo 4.
                \end{enumerate}
            \end{enumerate}       
        \end{tcolorbox}

        \begin{tcolorbox}[breakable, colback=white,colframe=black!80!white,title=\textbf{C6 - Rimuovi libro}]
            \textbf{Attori presenti:} Amministratore.\vspace{.2cm}     
                
            \textbf{Precondizioni:} L’amministratore si trova sulla schermata di visualizzazione e gestione dei libri;\\È presente almeno un libro nella lista dei libri;\\L'archivio è stato caricato correttamente dal file all'avvio del sistema.\vspace{.2cm}
                
            \textbf{Postcondizioni:} Il libro selezionato viene rimosso con successo dalla lista dei libri.\vspace{.2cm}

            \textbf{Flusso di eventi normale:}\vspace{-0.2cm}
            \begin{enumerate}
                \setlength\itemsep{-0.3em}
                \item L’amministratore seleziona un libro presente nella lista dei libri;
                \item L’amministratore preme il pulsante “Rimuovi”;
                \item Viene mostrata una nuova finestra con un messaggio di avviso;
                \item L’amministratore conferma l’operazione premendo il pulsante “Si”;
                \item Il sistema rimuove i dati del libro dall’archivio;
                \item La finestra di conferma viene chiusa;
                \item L’amministratore visualizza la lista dei libri aggiornata. 
            \end{enumerate}

            \textbf{Flussi di eventi alternativi:}\vspace{-0.2cm}
            \begin{enumerate}
                \item[4a.] L'amministratore preme il pulsante "No";\vspace{-0.2cm}
                \begin{enumerate}
                    \item[4a.1] L'esecuzione riprende dal passo 1.
                \end{enumerate}
            \end{enumerate}       
        \end{tcolorbox}

        \begin{tcolorbox}[breakable, colback=white,colframe=black!80!white,title=\textbf{C7 - Inserisci utente}]
            \textbf{Attori presenti:} Amministratore.\vspace{.2cm}     
                
            \textbf{Precondizioni:} L’amministratore si trova sulla schermata di visualizzazione e gestione degli utenti.\vspace{.2cm}
                
            \textbf{Postcondizioni:} L'utente viene aggiunto con successo alla lista degli utenti.\vspace{.2cm}

            \textbf{Flusso di eventi normale:}\vspace{-0.2cm}
            \begin{enumerate}
                \setlength\itemsep{-0.3em}
                \item L’amministratore preme il pulsante “Aggiungi”;
                \item Viene mostrata una nuova finestra con i campi richiesti per l'inserimento (specificati in \hyperlink{DF-3}{DF-3}); 
                \item L’amministratore inserisce tutte le informazioni necessarie per l’inserimento dell'utente;
                \item L’amministratore preme il pulsante “Conferma”;
                \item Il sistema convalida i dati (formato corretto e campi obbligatori compilati);
                \item Il sistema verifica l’univocità della matricola;
                \item Il sistema inserisce il nuovo utente all’interno dell’archivio;
                \item La finestra di inserimento dei campi viene chiusa;
                \item L’amministratore visualizza la lista degli utenti aggiornata. 
            \end{enumerate}

            \textbf{Flussi di eventi alternativi:}\vspace{-0.2cm}
            \begin{enumerate}
                \setlength\itemsep{-0.3em}
                \item[4a.] L’amministratore preme il pulsante “Annulla”;\vspace{-0.2cm}
                \begin{enumerate}
                    \item[4a.1] L’esecuzione riprende dal passo 1.
                \end{enumerate}
                \item[5a.] Il sistema riscontra degli errori nel formato dei campi inseriti;\vspace{-0.2cm}
                \begin{enumerate}
                    \setlength\itemsep{-0.3em}
                    \item[5a.1] Viene visualizzato un messaggio di errore e vengono evidenziati i campi da correggere;
                    \item[5a.2] L’esecuzione riprende dal passo 3.
                \end{enumerate}
                \item[5b.] Il sistema si accorge che non sono stati inseriti tutti i campi obbligatori;\vspace{-0.2cm}
                \begin{enumerate}
                    \setlength\itemsep{-0.3em}
                    \item[5b.1] Viene visualizzato un messaggio di errore e vengono evidenziati i campi da inserire;
                    \item[5b.2] L’esecuzione riprende dal passo 3.
                \end{enumerate}
                \item[6a.] La matricola inserita corrisponde ad un utente già presente nell’archivio;\vspace{-0.2cm}
                \begin{enumerate}
                    \setlength\itemsep{-0.3em}
                    \item[6a.1] Viene visualizzato un messaggio di errore;
                    \item[6a.2] L’esecuzione riprende dal passo 3.
                \end{enumerate}
            \end{enumerate}       
        \end{tcolorbox}

        \begin{tcolorbox}[breakable, colback=white,colframe=black!80!white,title=\textbf{C8 - Modifica utente}]
            \textbf{Attori presenti:} Amministratore.\vspace{.2cm}     
                
            \textbf{Precondizioni:} L’amministratore si trova sulla schermata di visualizzazione e gestione degli utenti;\\È presente almeno un utente nella lista degli utenti;\\L'archivio è stato caricato correttamente dal file all'avvio del sistema.\vspace{.2cm}
                
            \textbf{Postcondizioni:} Uno o più campi dell'utente selezionato vengono modificati con successo.\vspace{.2cm}

            \textbf{Flusso di eventi normale:}\vspace{-0.2cm}
            \begin{enumerate}
                \setlength\itemsep{-0.3em}
                \item L’amministratore seleziona un libro presente nella lista degli utenti;
                \item L’amministratore preme il pulsante “Modifica”;
                \item Viene mostrata una nuova finestra contenente i campi correnti dell'utente selezionato;
                \item L’amministratore modifica uno o più campi;
                \item L’amministratore preme il pulsante “Conferma”.
                \item Il sistema convalida i dati (formato corretto e campi obbligatori compilati);
                \item Il sistema aggiorna i dati dell'utente nell’archivio;
                \item La finestra di modifica dei campi viene chiusa;
                \item L’amministratore visualizza la lista degli utenti aggiornata. 
            \end{enumerate}

            \textbf{Flussi di eventi alternativi:}\vspace{-0.2cm}
            \begin{enumerate}
                \setlength\itemsep{-0.3em}
                \item[5a.] L’amministratore preme il pulsante “Annulla”;\vspace{-0.2cm}
                \begin{enumerate}
                    \item[5a.1] L’esecuzione riprende dal passo 1.
                \end{enumerate}
                \item[6a.] Il sistema riscontra degli errori nel formato dei campi inseriti;\vspace{-0.2cm}
                \begin{enumerate}
                    \setlength\itemsep{-0.3em}
                    \item[6a.1] Viene visualizzato un messaggio di errore e vengono evidenziati i campi da correggere;
                    \item[6a.2] L’esecuzione riprende dal passo 4.
                \end{enumerate}
                \item[6b.] Il sistema si accorge che non sono stati inseriti tutti i campi obbligatori;\vspace{-0.2cm}
                \begin{enumerate}
                    \setlength\itemsep{-0.3em}
                    \item[6b.1] Viene visualizzato un messaggio di errore e vengono evidenziati i campi da inserire;
                    \item[6b.2] L’esecuzione riprende dal passo 4.
                \end{enumerate}
            \end{enumerate}       
        \end{tcolorbox}

        \begin{tcolorbox}[breakable, colback=white,colframe=black!80!white,title=\textbf{C9 - Rimuovi utente}]
            \textbf{Attori presenti:} Amministratore.\vspace{.2cm}     
                
            \textbf{Precondizioni:} L’amministratore si trova sulla schermata di visualizzazione e gestione degli utenti;\\È presente almeno un utente nella lista degli utenti;\\L'archivio è stato caricato correttamente dal file all'avvio del sistema.\vspace{.2cm}
                
            \textbf{Postcondizioni:} L'utente selezionato viene rimosso con successo dalla lista degli utenti.\vspace{.2cm}

            \textbf{Flusso di eventi normale:}\vspace{-0.2cm}
            \begin{enumerate}
                \setlength\itemsep{-0.3em}
                \item L’amministratore seleziona un utente presente nella lista degli utenti;
                \item L’amministratore preme il pulsante “Rimuovi”;
                \item Viene mostrata una nuova finestra con un messaggio di avviso;
                \item L’amministratore conferma l’operazione premendo il pulsante “Si”;
                \item Il sistema rimuove i dati dell'utente dall’archivio;
                \item La finestra di conferma viene chiusa;
                \item L’amministratore visualizza la lista degli utenti aggiornata. 
            \end{enumerate}

            \textbf{Flussi di eventi alternativi:}\vspace{-0.2cm}
            \begin{enumerate}
                \item[4a.] L'amministratore preme il pulsante "No";\vspace{-0.2cm}
                \begin{enumerate}
                    \item[4a.1] L'esecuzione riprende dal passo 1.
                \end{enumerate}
            \end{enumerate}         
        \end{tcolorbox}

        \begin{tcolorbox}[breakable, colback=white,colframe=black!80!white,title=\textbf{C10 - Ricerca libro}]
            \textbf{Attori presenti:} Amministratore.\vspace{.2cm}     
                
            \textbf{Precondizioni:} L’amministratore si trova sulla schermata di visualizzazione e gestione dei libri;\\L'archivio è stato caricato correttamente dal file all'avvio del sistema.\vspace{.2cm}
                
            \textbf{Postcondizioni:} Vengono mostrati soltanto i libri che soddisfano il criterio di ricerca;\\I libri vengono visualizzati in ordine alfabetico, per titolo.\vspace{.2cm}

            \textbf{Flusso di eventi normale:}\vspace{-0.2cm}
            \begin{enumerate}
                \setlength\itemsep{-0.3em}
                \item L’amministratore seleziona la barra di ricerca;
                \item L’amministratore inserisce una sottostringa del titolo, del nome/cognome di uno degli autori o dell’ISBN del libro da ricercare all'interno della barra di ricerca;
                \item Il sistema effettua in tempo reale la ricerca;
                \item Vengono visualizzati soltanto i libri che hanno almeno una corrispondenza con la sottostringa inserita.
            \end{enumerate}

            \textbf{Flussi di eventi alternativi:}\vspace{-0.2cm}
            \begin{enumerate}
                \setlength\itemsep{-0.3em}
                \item[2a.] L’amministratore cancella la sottostringa dalla barra di ricerca;\vspace{-0.2cm}
                \begin{enumerate}
                    \setlength\itemsep{-0.3em}
                    \item[2a.1] Viene ripristinata la visualizzazione della lista completa dei libri.
                \end{enumerate}
                \item[3a.] Il sistema non trova alcuna corrispondenza per la sottostringa inserita;\vspace{-0.2cm}
                \begin{enumerate}
                    \setlength\itemsep{-0.3em}
                    \item[3a.1] Viene visualizzata una tabella vuota con un messaggio informativo;
                    \item[3a.2] L’esecuzione riprende dal passo 2.
                \end{enumerate}
            \end{enumerate}       
        \end{tcolorbox}

        \begin{tcolorbox}[breakable, colback=white,colframe=black!80!white,title=\textbf{C12 - Ricerca utente}]
            \textbf{Attori presenti:} Amministratore.\vspace{.2cm}     
                
            \textbf{Precondizioni:} L’amministratore si trova sulla schermata di visualizzazione e gestione degli utenti;\\L'archivio è stato caricato correttamente dal file all'avvio del sistema.\vspace{.2cm}
                
            \textbf{Postcondizioni:} Vengono mostrati soltanto gli utenti che soddisfano il criterio di ricerca;\\Gli utenti vengono visualizzati in ordine alfabetico, per cognome e (a parità di cognome) per nome.\vspace{.2cm}

            \textbf{Flusso di eventi normale:}\vspace{-0.2cm}
            \begin{enumerate}
                \setlength\itemsep{-0.3em}
                \item L’amministratore seleziona la barra di ricerca;
                \item L’amministratore inserisce una sottostringa del cognome o della matricola dell'utente da ricercare all'interno della barra di ricerca;
                \item Il sistema effettua in tempo reale la ricerca;
                \item Vengono visualizzati soltanto gli utenti che hanno almeno una corrispondenza con la sottostringa inserita.
            \end{enumerate}

            \textbf{Flussi di eventi alternativi:}\vspace{-0.2cm}
            \begin{enumerate}
                \setlength\itemsep{-0.3em}
                \item[2a.] L’amministratore cancella la sottostringa dalla barra di ricerca;\vspace{-0.2cm}
                \begin{enumerate}
                    \setlength\itemsep{-0.3em}
                    \item[2a.1] Viene ripristinata la visualizzazione della lista completa degli utenti.
                \end{enumerate}
                \item[3a.] Il sistema non trova alcuna corrispondenza per la sottostringa inserita;\vspace{-0.2cm}
                \begin{enumerate}
                    \setlength\itemsep{-0.3em}
                    \item[3a.1] Viene visualizzata una tabella vuota con un messaggio informativo;
                    \item[3a.2] L’esecuzione riprende dal passo 2.
                \end{enumerate}
            \end{enumerate}       
        \end{tcolorbox}

        \begin{tcolorbox}[breakable, colback=white,colframe=black!80!white,title=\textbf{C13 - Registra prestito}]
            \textbf{Attori presenti:} Amministratore.\vspace{.2cm}     
                
            \textbf{Precondizioni:} L’amministratore si trova sulla schermata di visualizzazione e gestione dei prestiti;\\L'archivio è stato caricato correttamente dal file all'avvio del sistema.\vspace{.2cm}
                
            \textbf{Postcondizioni:} Il prestito viene aggiunto con successo alla lista dei prestiti;\\Il numero di copie disponibili del libro prestato viene aggiornato dal sistema;\\La lista dei libri presi in prestito dall’utente viene aggiornata dal sistema.\vspace{.2cm}

            \textbf{Flusso di eventi normale:}\vspace{-0.2cm}
            \begin{enumerate}
                \setlength\itemsep{-0.3em}
                \item L’amministratore preme il pulsante “Inserisci”;
                \item Viene mostrata una nuova finestra con i campi richiesti per l'inserimento (specificati in \hyperlink{DF-4}{DF-4});
                \item L’amministratore inserisce tutte le informazioni necessarie per la registrazione del prestito;
                \item L’amministratore preme il pulsante “Conferma”;
                \item Il sistema verifica l'esistenza dell'utente e del libro associati ai dati inseriti;
                \item Il sistema convalida i dati (formato corretto, campi obbligatori compilati e data logicamente corretta);
                \item Il sistema inserisce il nuovo prestito all’interno dell’archivio;
                \item Il sistema decrementa il numero di copie disponibili del libro prestato e lo aggiunge alla lista dei libri presi in prestito dall’utente;
                \item La finestra di inserimento dei campi viene chiusa;
                \item L’amministratore visualizza la lista dei prestiti aggiornata. 
            \end{enumerate}

            \textbf{Flussi di eventi alternativi:}\vspace{-0.2cm}
            \begin{enumerate}
                \item[4a.] L'amministratore preme il pulsante "Annulla";\vspace{-0.2cm}
                \begin{enumerate}
                    \item[4a.1] L'esecuzione riprende dal passo 1.
                \end{enumerate}
                \item[5a.] Il sistema si accorge dell'inesistenza dell'utente e/o del libro all'interno dell'archivio;\vspace{-0.2cm}
                \begin{enumerate}
                    \item[5a.1] Viene visualizzato un messaggio di errore;
                    \item[5a.2] L’esecuzione riprende dal passo 3.
                \end{enumerate}
                \item[5b.] Il sistema si accorge che l'utente ha raggiunto il limite massimo di prestiti;\vspace{-0.2cm}
                \begin{enumerate}
                    \item[5b.1] Viene visualizzato un messaggio di errore;
                    \item[5b.2] L’esecuzione riprende dal passo 3.
                \end{enumerate}
                \item[5c.] Il sistema si accorge che il libro non ha copie disponibili;\vspace{-0.2cm}
                \begin{enumerate}
                    \item[5c.1] Viene visualizzato un messaggio di errore;
                    \item[5c.2] L’esecuzione riprende dal passo 3.
                \end{enumerate}
                \item[6a.] Il sistema riscontra degli errori nel formato dei campi inseriti;\vspace{-0.2cm}
                \begin{enumerate}
                    \item[6a.1] Viene visualizzato un messaggio di errore e vengono evidenziati i 		          campi da correggere;
                    \item[6a.2] L’esecuzione riprende dal passo 3.
                \end{enumerate}
                \item[6b.] Il sistema si accorge che non sono stati inseriti tutti i campi obbligatori;\vspace{-0.2cm}
                \begin{enumerate}
                    \item[6b.1] Viene visualizzato un messaggio di errore e vengono evidenziati i campi da inserire;
                    \item[6b.2] L'esecuzione riprende dal passo 3.
                \end{enumerate}
            \end{enumerate}       
        \end{tcolorbox}

        \begin{tcolorbox}[breakable, colback=white,colframe=black!80!white,title=\textbf{C14 - Registra restituzione}]
            \textbf{Attori presenti:} Amministratore.\vspace{.2cm}     
                
            \textbf{Precondizioni:} L’amministratore si trova sulla schermata di visualizzazione e gestione dei prestiti;\\È presente almeno un prestito nella lista dei prestiti;\\L'archivio è stato caricato correttamente dal file all'avvio del sistema.\vspace{.2cm}
                
            \textbf{Postcondizioni:} La restituzione del libro viene registrata con successo;\\Il numero di copie disponibili del libro prestato viene aggiornato dal sistema;\\La lista dei libri presi in prestito dall’utente viene aggiornata dal sistema.\vspace{.2cm}

            \textbf{Flusso di eventi normale:}\vspace{-0.2cm}
            \begin{enumerate}
                \setlength\itemsep{-0.3em}
                \item L’amministratore seleziona un prestito presente nella lista dei prestiti;
                \item L’amministratore preme il pulsante “Restituzione”;
                \item Viene mostrata una nuova finestra con un messaggio di avviso;
                \item L’amministratore conferma l’operazione premendo il pulsante “Sì”;
                \item Il sistema registra l’avvenuta restituzione;
                \item Il sistema incrementa il numero di copie disponibili del libro prestato e lo rimuove dalla lista dei libri presi in prestito dall’utente;
                \item La finestra di conferma viene chiusa;
                \item L’amministratore visualizza la lista dei prestiti aggiornata. 
            \end{enumerate}

            \textbf{Flussi di eventi alternativi:}\vspace{-0.2cm}
            \begin{enumerate}
                \item[4a.] L'amministratore preme il pulsante "No";\vspace{-0.2cm}
                \begin{enumerate}
                    \item[4a.1] L'esecuzione riprende dal passo 1.
                \end{enumerate}
            \end{enumerate}       
        \end{tcolorbox}

        \begin{tcolorbox}[breakable, colback=white,colframe=black!80!white,title=\textbf{C15 - Estendi prestito}]
            \textbf{Attori presenti:} Amministratore.\vspace{.2cm}     
                
            \textbf{Precondizioni:} L’amministratore si trova sulla schermata di visualizzazione e gestione dei prestiti;\\È presente almeno un prestito nella lista dei prestiti;\\L'archivio è stato caricato correttamente dal file all'avvio del sistema.\vspace{.2cm}
                
            \textbf{Postcondizioni:} La data prevista di restituzione del prestito selezionato viene aggiornata con successo.\vspace{.2cm}

            \textbf{Flusso di eventi normale:}\vspace{-0.2cm}
            \begin{enumerate}
                \setlength\itemsep{-0.3em}
                \item L’amministratore seleziona un prestito presente nella lista dei prestiti;
                \item L’amministratore preme il pulsante “Estendi”;
                \item Viene mostrata una nuova finestra con i campi correnti del prestito selezionato;
                \item L’amministratore modifica la data prevista di restituzione del prestito;
                \item L’amministratore preme il pulsante “Conferma”;
                \item Il sistema convalida il dato (formato corretto, campo obbligatorio compilato e data logicamente corretta);
                \item Il sistema aggiorna i dati del prestito nell’archivio;
                \item La finestra di modifica del campo viene chiusa;
                \item L’amministratore visualizza la lista dei prestiti aggiornata. 
            \end{enumerate}

            \textbf{Flussi di eventi alternativi:}\vspace{-0.2cm}
            \begin{enumerate}
                \item[5a.] L'amministratore preme il pulsante "Annulla";\vspace{-0.2cm}
                \begin{enumerate}
                    \item[5a.1] L'esecuzione riprende dal passo 1.
                \end{enumerate}
                \item[6a.] Il sistema riscontra degli errori nel formato del campo inserito;\vspace{-0.2cm}
                \begin{enumerate}
                    \item[6a.1] Viene visualizzato un messaggio di errore e viene evidenziato il campo da correggere;
                    \item[6a.2] L’esecuzione riprende dal passo 4.
                \end{enumerate}
                \item[6b.] Il sistema si accorge che il campo è stato lasciato vuoto;\vspace{-0.2cm}
                \begin{enumerate}
                    \item[6b.1] Viene visualizzato un messaggio di errore e viene evidenziato il campo da inserire;
                    \item[6b.2] L'esecuzione riprende dal passo 4.
                \end{enumerate}
                \item[6c.] Il sistema si accorge che è stata inserita una data antecedente alla data odierna oppure alla data prevista di restituzione precedente;\vspace{-0.2cm}
                \begin{enumerate}
                    \item[6c.1] Viene visualizzato un messaggio di errore e viene evidenziato il campo da modificare;
                    \item[6c.2] L'esecuzione riprende dal passo 4.
                \end{enumerate}
            \end{enumerate}       
        \end{tcolorbox}

        \begin{tcolorbox}[breakable, colback=white,colframe=black!80!white,title=\textbf{C16 - Filtra lista prestiti}]
            \textbf{Attori presenti:} Amministratore.\vspace{.2cm}     
                
            \textbf{Precondizioni:} L’amministratore si trova sulla schermata di visualizzazione e gestione dei prestiti;\\L'archivio è stato caricato correttamente dal file all'avvio del sistema.\vspace{.2cm}
                
            \textbf{Postcondizioni:} Vengono mostrati soltanto i prestiti che soddisfano il criterio di filtraggio selezionato;\\I prestiti vengono visualizzati in ordine cronologico, per data prevista di restituzione.\vspace{.2cm}

            \textbf{Flusso di eventi normale:}\vspace{-0.2cm}
            \begin{enumerate}
                \setlength\itemsep{-0.3em}
                \item L’amministratore seleziona la voce "ATTIVI" dall'elenco dei filtri;
                \item Vengono visualizzati soltanto i prestiti che soddisfano il criterio applicato.
            \end{enumerate}

            \textbf{Flussi di eventi alternativi:}\vspace{-0.2cm}
            \begin{enumerate}
                \setlength\itemsep{-0.3em}
                \item[1a.] L’amministratore seleziona la voce "CONCLUSI" dall'elenco dei filtri;
                    \begin{enumerate}
                    \setlength\itemsep{-0.3em}
                    \item[1a.1] L'esecuzione riprende dal passo 2.
                \end{enumerate}
                \item[1b.] L’amministratore seleziona la voce "TUTTI" dall'elenco dei filtri; 
                \begin{enumerate}
                    \setlength\itemsep{-0.3em}
                    \item[1b.1] Viene ripristinata la visualizzazione della lista completa dei prestiti.
                \end{enumerate}
            \end{enumerate}       
        \end{tcolorbox}
        
\end{document}
