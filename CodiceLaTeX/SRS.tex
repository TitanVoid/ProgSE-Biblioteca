\documentclass[12pt, a4paper]{article}

\usepackage[top=2.3cm, bottom=2cm, left=2cm, right=2cm, headheight=15pt]{geometry}
\usepackage{graphicx}
\usepackage{svg}
\usepackage[italian]{babel}
\usepackage{lastpage}
\usepackage{adjustbox}
\usepackage{enumitem}
\usepackage[most]{tcolorbox}
%\usepackage{xcolor}
\usepackage[colorlinks=true]{hyperref}
\usepackage{fancyhdr}
\pagestyle{fancy}
\usepackage{caption}
\renewcommand{\contentsname}{Indice}
\renewcommand{\sectionmark}[1]{\markboth{#1}{#1}}
\graphicspath{{images/}} % COSE DA FARE: Directory "images" per le foto da inserire nel documento!


\begin{document}
    \begin{titlepage}
        \fancyhf{}
		\centering
		
		{\Large \textsc{Ingegneria del Software - A.A. 2025/2026}}\\[.4cm]
        {\Large Applicazione per la gestione di una Biblioteca universitaria}\\[4cm]
		
		{\huge \textbf{Documento di Specifica dei\\Requisiti - SRS}}\\[12.5cm]
		
		\large { 
			\textit{\emph{Gruppo 9:}\\
            Luisa Genovese\\
            Erica Brancaccio\\
            Paolo Alfé\\
            Francesco Altieri}
		}
		
		\vspace{1.5cm}
		
		{\large \today}
		
		
	\end{titlepage}
    
    \fancyhf{}
    \fancyhead[R]{pagina \thepage\ di \pageref*{LastPage}}
    \pagestyle{fancy}
    \hypersetup{linkcolor=black}
    \tableofcontents
    \newpage

    \section*{Descrizione introduttiva}
    \addcontentsline{toc}{section}{Descrizione introduttiva}
        \subsection*{Obiettivo}
            Si vuole realizzare un software per la gestione di una biblioteca universitaria.\vspace{.2cm}
            \\Questo vuole garantire all’utente, in modo semplice ed intuitivo, la totale gestione di una lista di libri (intesa come aggiunta, cancellazione e modifica di un libro presente all’interno dell’archivio).
            \\Inoltre all’utente, mediante l'utilizzo di questo software, è garantita la gestione di tutti gli utenti che hanno richiesto dei prestiti, tenendo traccia dei tempi di restituzione ed evidenziando eventuali ritardi.\vspace{.2cm}
            \\Si rende disponibile un’interfaccia grafica che permette all’utente di usufruire delle funzionalità offerte dal sistema.
        \subsection*{Stakeholder}
            

    \newpage
    \fancyhf{}
    \fancyhead[R]{pagina \thepage\ di \pageref*{LastPage}}
    \fancyhead[L]{\nouppercase{\leftmark}}
    \section{Ingegneria dei Requisiti - SRS}
    \subsection{Tabella di categorizzazione dei requisiti}
        \begin{adjustbox}{width=.8\paperwidth, height=.77\paperheight, center, margin=0cm 0cm 0cm .5cm}
            \includegraphics{images/categorizzazione_requisiti1.png}
        \end{adjustbox}
        
        \newpage
        \begin{figure}[h!]
            \begin{adjustbox}{width=.8\paperwidth, height=.78\paperheight, center, margin=0cm 0cm 0cm 1.4cm}
                \includegraphics{images/categorizzazione_requisiti2.png}
            \end{adjustbox}
        \end{figure}

        \newpage
        \begin{figure}[h!]
            \begin{adjustbox}{width=.8\paperwidth, height=.21\paperheight, margin=0cm 0cm 0cm .5cm}
                \includegraphics{images/categorizzazione_requisiti3.png}
            \end{adjustbox}
            \caption{Tabella di categorizzazione dei requisiti}
        \end{figure}

    \newpage
    \hypersetup{linkcolor=blue}
    \subsection{Requisiti Funzionali}
        \begin{tcolorbox}[breakable, colback=white,colframe=black!80!white,title=\textbf{Funzionalità individuali - IF}]
            \begin{itemize}[itemsep=6pt]
                \hypertarget{IF-1}{\item[\textbf{IF-1}]}\textbf{Inserimento di un libro}
                \\L’applicazione consente all’amministratore di inserire un libro all’interno della lista dei libri.\vspace{.2cm}
                \\Premendo l’apposito pulsante, comparirà una finestra aggiuntiva in cui si dovranno inserire tutti i campi necessari, coerentemente a quanto definito in \hyperlink{DF-2}{DF-2}. 
                \\Al termine dell’inserimento, sarà possibile confermare oppure annullare l’operazione mediante i rispettivi pulsanti. 
                \\Nel caso in cui non siano stati inseriti tutti i campi necessari, oppure nel caso di inserimento di campi errati (vedere \hyperlink{DF-2}{DF-2} per le informazioni relative al formato dei dati), non sarà possibile confermare l’inserimento. 
                \\Se il codice identificativo univoco fornito in input dovesse corrispondere ad un libro già presente nella lista, l'operazione di inserimento sarà bloccata e l’amministratore verrà notificato dell’esistenza di un duplicato. 
                      
                \hypertarget{IF-2}{\item[\textbf{IF-2}]}\textbf{Modifica di un libro}
                \\L’applicazione consente all’amministratore di modificare le informazioni associate ad un libro.\vspace{.2cm}
                \\Selezionando un libro dalla lista dei libri e premendo l’apposito pulsante, sarà possibile apportare modifiche ad uno o più dei suoi campi, ad eccezione del campo “Codice Identificativo”.
                \\Sarà possibile salvare le modifiche effettuate oppure annullare l’operazione tramite appositi pulsanti.
                \\Il sistema dovrà impedire il salvataggio delle modifiche qualora uno o più dei campi (definiti in \hyperlink{DF-2}{DF-2}) dovessero essere lasciati vuoti. 

                \hypertarget{IF-3}{\item[\textbf{IF-3}]}\textbf{Cancellazione di un libro}
                \\L’applicazione consente all’amministratore di rimuovere un libro dalla lista dei libri.\vspace{.2cm}
                \\L'operazione viene avviata selezionando un libro presente nella lista dei libri e premendo l’apposito pulsante di cancellazione.
                \\Il sistema mostrerà un messaggio di conferma all’amministratore, offrendogli la possibilità di portare a termine o annullare l’operazione tramite i rispettivi pulsanti.
                \\La rimozione di un libro dalla lista dei libri non deve comportare la cancellazione dello storico dei prestiti precedentemente associati a quel volume, né annullare gli eventuali prestiti attivi al momento della rimozione.
                \\L'operazione, una volta confermata, è irreversibile. 

                \hypertarget{IF-4}{\item[\textbf{IF-4}]}\textbf{Visualizzazione lista libri per titolo}
                \\La visualizzazione dei libri contenuti nella lista dei libri avviene sempre in ordine alfabetico ascendente (A-Z), rispetto al loro titolo.\vspace{.2cm}
                \\L’ordinamento della lista dovrà essere mantenuto dopo ogni operazione di aggiunta, modifica o rimozione di un libro presente nella lista dei libri. 

                \hypertarget{IF-5}{\item[\textbf{IF-5}]}\textbf{Ricerca di un libro}
                \\L’applicazione consente all’amministratore di ricercare un libro all’interno della lista dei libri.\vspace{.2cm}
                \\L’operazione potrà avvenire digitando nella barra di ricerca uno tra i seguenti campi del libro da ricercare: 
                \begin{itemize}[label=$\bullet$]
                    \item Una sottostringa del titolo; 
                    \item Una sottostringa del nome dell’autore;
                    \item Una sottostringa del cognome dell'autore;
                    \item Una sottostringa del codice identificativo. 
                \end{itemize}
                La corrispondenza dovrà ignorare la distinzione tra maiuscole e minuscole.
                \\Qualora la ricerca non dovesse produrre alcuna corrispondenza, all'amministratore verrà presentata una tabella vuota.
                \\In caso di risultati multipli, i libri restituiti verranno mostrati secondo il criterio d’ordinamento definito in \hyperlink{IF-4}{IF-4}. 

                \hypertarget{IF-6}{\item[\textbf{IF-6}]}\textbf{Inserimento di un utente}
                \\L’applicazione consente all’amministratore di inserire un utente all’interno della lista dei libri.\vspace{.2cm}
                \\Premendo l’apposito pulsante, comparirà una finestra aggiuntiva in cui si dovranno inserire tutti i campi necessari, coerentemente a quanto definito in \hyperlink{DF-3}{DF-3}.
                \\Al termine dell’inserimento, sarà possibile confermare oppure annullare l’operazione mediante i rispettivi pulsanti.
                \\Nel caso in cui non siano stati inseriti tutti i campi necessari, oppure nel caso di inserimento di campi errati (vedere \hyperlink{DF-3}{DF-3} per le informazioni relative al formato dei dati), non sarà possibile confermare l’inserimento.
                \\Se la matricola fornita in input dovesse corrispondere ad un utente già presente nella lista, l'operazione di inserimento sarà bloccata e l’amministratore verrà notificato dell’esistenza di un duplicato. 
                \begin{itemize} 
                    \hypertarget{IF-6.1}{\item[\textbf{IF-6.1}]}\textbf{Inserimento del numero di telefono di un utente}
                    \\L’applicazione consente all’amministratore di specificare come ulteriore informazione il numero telefonico di un utente, all’atto del suo inserimento nella lista degli utenti (vedere \hyperlink{IF-6}{IF-6}).\vspace{.2cm}
                    \\Il suo formato dovrà essere conforme a quanto definito in \hyperlink{DF-3.1}{DF-3.1}, altrimenti non sarà possibile confermare l’inserimento. 
                \end{itemize}
                
                \hypertarget{IF-7}{\item[\textbf{IF-7}]}\textbf{Modifica di un utente}
                \\L’applicazione consente all’amministratore di modificare le informazioni associate ad un utente.\vspace{.2cm}
                \\Selezionando un utente dalla lista degli utenti e premendo l’apposito pulsante, sarà possibile apportare modifiche ad uno o più dei suoi campi, ad eccezione del campo “Matricola”.
                \\Sarà possibile salvare le modifiche effettuate oppure annullare l’operazione tramite appositi pulsanti.
                \\Il sistema dovrà impedire il salvataggio delle modifiche qualora uno o più dei campi (definiti in \hyperlink{DF-3}{DF-3}) dovessero essere lasciati vuoti. 

                \hypertarget{IF-8}{\item[\textbf{IF-8}]}\textbf{Cancellazione di un utente}
                \\L’applicazione consente all’amministratore di rimuovere un utente dalla lista degli utenti.\vspace{.2cm}
                \\L'operazione viene avviata selezionando un utente presente nella lista degli utenti e premendo l’apposito pulsante di cancellazione.
                \\Il sistema mostrerà un messaggio di conferma all’amministratore, offrendogli la possibilità di portare a termine o annullare l’operazione tramite i rispettivi pulsanti.
                \\La rimozione di un utente dalla lista degli utenti non deve comportare la cancellazione dello storico dei prestiti precedentemente associati a quell'utente né annullare gli eventuali prestiti attivi al momento della rimozione.
                \\L'operazione, una volta confermata, è irreversibile. 

                \hypertarget{IF-9}{\item[\textbf{IF-9}]}\textbf{Visualizzazione degli utenti per cognome e nome}
                \\La visualizzazione degli utenti contenuti nella lista degli utenti avviene sempre in ordine alfabetico ascendente (A-Z), rispetto al loro cognome e nome.\vspace{.2cm}
                \\L’ordinamento della lista dovrà essere mantenuto dopo ogni operazione di aggiunta, modifica o rimozione di un utente presente nella lista degli utenti. 

                \hypertarget{IF-10}{\item[\textbf{IF-10}]}\textbf{Ricerca di un utente}
                \\L’applicazione consente all’amministratore di ricercare un utente all’interno della lista degli utenti.\vspace{.2cm}
                \\L’operazione potrà avvenire digitando nella barra di ricerca uno tra i seguenti campi dell’utente da ricercare: 
                \begin{itemize}[label=$\bullet$]
                    \item Una sottostringa del cognome; 
                    \item Una sottostringa della matricola. 
                \end{itemize}
                La corrispondenza dovrà ignorare la distinzione tra maiuscole e minuscole.
                \\Qualora la ricerca non dovesse produrre alcuna corrispondenza, all'amministratore verrà presentata una tabella vuota.
                \\In caso di risultati multipli, gli utenti restituiti verranno mostrati secondo il criterio d’ordinamento definito in \hyperlink{IF-9}{IF-9}. 

                \hypertarget{IF-11}{\item[\textbf{IF-11}]}\textbf{Registrazione di un prestito}
                \\L’applicazione consente all’amministratore di registrare un nuovo prestito.\vspace{.2cm}
                \\Premendo l’apposito pulsante, comparirà una finestra aggiuntiva in cui si dovranno inserire tutti i campi necessari, coerentemente a quanto definito in \hyperlink{DF-4}{DF-4}.
                \\Al termine dell’inserimento, sarà possibile confermare oppure annullare l’operazione mediante i rispettivi pulsanti.
                \\All’atto di una richiesta di prestito, il sistema dovrà svolgere automaticamente le seguenti operazioni:
                \begin{itemize}[label=$\bullet$]
                    \item Aggiornamento del numero di copie disponibili del libro prestato; 
                    \item Aggiornamento della lista dei libri associati all’utente, aggiungendo il volume appena preso in prestito; 
                    \item Aggiunta del prestito dalla schermata di visualizzazione dei prestiti attivi.
                \end{itemize}
                Nel caso in cui non siano stati inseriti tutti i campi necessari, oppure nel caso di inserimento di campi errati (vedere \hyperlink{DF-4}{DF-4} per le informazioni relative al formato dei dati), non sarà possibile confermare la registrazione.
                \\Se il codice identificativo del libro e/o la matricola dell’utente forniti in input non dovessero essere presenti nelle rispettive liste, l'operazione di registrazione sarà bloccata. 

                \hypertarget{IF-12}{\item[\textbf{IF-12}]}\textbf{Visualizzazione dei prestiti attivi per data prevista di restituzione}
                \\La visualizzazione dei prestiti attivi contenuti nella lista dei prestiti avviene sempre in ordine cronologico crescente, rispetto alla data di restituzione.\vspace{.2cm}
                \\L’ordinamento della lista dovrà essere mantenuto dopo ogni operazione di registrazione, di estensione di un prestito o di restituzione di un libro. 

                \hypertarget{IF-13}{\item[\textbf{IF-13}]}\textbf{Registrazione di una restituzione}
                \\L’applicazione permette all’amministratore di registrare la restituzione di un libro precedentemente preso in prestito.\vspace{.2cm}
                \\L’amministratore dovrà selezionare, dalla lista dei prestiti attivi, il prestito da chiudere e avviare l’operazione di restituzione premendo l’apposito pulsante.
                \\All’atto di una restituzione, il sistema dovrà svolgere automaticamente le seguenti operazioni: 
                \begin{itemize}[label=$\bullet$]
                    \item Aggiornamento del numero di copie disponibili del libro restituito;
                    \item Aggiornamento della lista dei libri associati all’utente, rimuovendo il volume appena restituito;
                    \item Rimozione del prestito dalla schermata di visualizzazione dei prestiti attivi.
                \end{itemize}
                Tutti i prestiti conclusi saranno comunque archiviati all’interno di un file, coerentemente a quanto definito in \hyperlink{DF-1}{DF-1}. 

                \hypertarget{IF-14}{\item[\textbf{IF-14}]}\textbf{Ricerca rapida mediante filtro}
                \\L’applicazione supporta, nel caso della lista dei libri, una funzionalità di ricerca avanzata mediante criteri di filtro multipli e combinati.\vspace{.2cm}
                \\Tali criteri di filtro potranno essere applicati autonomamente, oppure in combinazione con la ricerca testuale tradizionale (definita in \hyperlink{IF-5}{IF-5}).
                \\Dovrà essere supportare la possibilità di filtrare la lista dei libri sui seguenti campi: 
                \begin{itemize}[label=$\bullet$]
                    \item Anno di pubblicazione: sarà possibile visualizzare tutti i libri pubblicati a partire da/prima di un determinato anno, oppure in un intervallo;
                    \item Disponibilità: potranno essere visualizzati tutti i libri il cui stato risulta essere “Disponibile” oppure “Non Disponibile”;
                    \item Numero di copie disponibili: sarà possibile visualizzare i libri che hanno una disponibilità maggiore/minore di un determinato numero di copie, oppure compresa in un intervallo.
                \end{itemize}
                I risultati della ricerca dovranno essere presentati rispettando l'ordinamento definito in \hyperlink{IF-4}{IF-4}. 

                \hypertarget{IF-15}{\item[\textbf{IF-15}]}\textbf{Estensione prestito}
                \\L’applicazione consente di estendere la data di restituzione prevista per un prestito attivo.\vspace{.2cm}
                \\In caso di esplicita richiesta da parte dell’utente, l’amministratore dovrà poter selezionare il prestito di interesse e modificarne il campo relativo alla data di restituzione (rispettandone il formato, definito in \hyperlink{DF-4}{DF-4}).
                \\L’estensione sarà soggetta a delle limitazioni temporali e numeriche: 
                \begin{itemize}[label=$\bullet$]
                    \item Il servizio potrà essere richiesto dagli utenti soltanto nel caso in cui non sia stata superata la data di restituzione prevista;
                    \item Per ogni prestito è ammessa un’unica richiesta di estensione.
                \end{itemize}
            \end{itemize}
        \end{tcolorbox}

        \begin{tcolorbox}[breakable, colback=white,colframe=black!80!white,title=\textbf{Business Flow - BF}]
            \begin{itemize}[itemsep=6pt]
                \hypertarget{BF-1}{\item[\textbf{BF-1}]}\textbf{Segnalazione ritardi}
                \\L’applicazione permette di segnalare all'utente i prestiti che superano la data prevista di restituzione, attraverso la loro identificazione automatica e l'intervento manuale dell’amministratore.\vspace{.2cm}
                \\Un prestito è considerato “in ritardo” se la data corrente è successiva alla data prevista di restituzione.
                \\Ogni prestito per il quale avviene ciò viene segnalato in maniera visiva all’amministratore (coerentemente a quanto definito in \hyperlink{UI-4}{UI-4}).
                \\L’amministratore, attraverso l'informazione relativa al numero di telefono che viene memorizzata dal sistema (vedere \hyperlink{DF-3.1}{DF-3.1}), una volta presa visione del ritardo chiamerà l’utente che ha effettuato il prestito, notificandolo della mancata restituzione del libro.
                \\Nel caso in cui la chiamata sia stata effettuata con successo, l’amministratore annoterà l’avvenuta segnalazione sull’applicazione, tramite la selezione di un’apposita casella.   
                
                \hypertarget{BF-2}{\item[\textbf{BF-2}]}\textbf{Prestito di un libro}
                \\La biblioteca offre all’utente la possibilità di ottenere in prestito un libro presente nel catalogo.
                \\Tramite l’applicazione, l’amministratore potrà successivamente procedere con la registrazione del prestito solo se vengono rispettate le seguenti condizioni:
                \begin{itemize}[label=$\bullet$]
                    \item L’utente deve essere presente nell’elenco di utenti della biblioteca;
                    \item L’utente non deve avere più di tre prestiti attivi contemporaneamente;
                    \item Il libro richiesto deve avere un numero di copie disponibili maggiore di zero. 
                \end{itemize}
                \vspace{.2cm}All’atto della richiesta di prestito da parte dell’utente, l’amministratore dovrà dapprima verificare la sua presenza all’interno dell’elenco di utenti gestito dal sistema.
                \\Qualora l’utente non dovesse essere presente in tale elenco, quest’ultimo fornirà i propri dati personali all’amministratore, che lo inserirà mediante l’operazione definita in \hyperlink{IF-6}{IF-6}.
                \\Una volta ottenute (o in alternativa, verificate) le informazioni personali dell’utente, esso mostrerà all’amministratore il libro o i libri che intende prendere in prestito.
                \\Per ogni libro richiesto, l’amministratore dovrà: 
                \begin{itemize}[label=$\bullet$]
                    \item Verificare che esso sia presente nel catalogo mediante una ricerca (vedere \hyperlink{IF-5}{IF-5});
                    \item Verificare che il libro sia disponibile;
                    \item Verificare che l’utente non superi il limite massimo di prestiti.
                \end{itemize}
                Se tutte le condizioni sono soddisfatte, l'amministratore, tramite l'applicazione, registra l'avvenuto prestito all'Utente (come definito in \hyperlink{IF-11}{IF-11}).
                \\Se uno dei libri richiesti in prestito non è disponibile, l'Amministratore notificherà l'indisponibilità all'utente.
                \\Se l'utente ha già tre prestiti attivi, l’amministratore notificherà all'utente la necessità di restituire i libri in suo possesso prima di poterne prendere altri in prestito. 
                
                \hypertarget{BF-3}{\item[\textbf{BF-3}]}\textbf{Restituzione di un libro}
                \\La biblioteca offre all’utente la possibilità di restituire un libro preso precedentemente in prestito.\vspace{.2cm}
                \\Tramite l’applicazione, l’amministratore potrà successivamente procedere con la registrazione dell'avvenuta restituzione.
                \\All’atto della richiesta di restituzione, l’utente dovrà identificarsi all’amministratore e presentare il libro o i libri che intende restituire.
                \\Una volta ottenuti i dati necessari, l’amministratore verificherà l’esistenza di un prestito attivo che corrisponde alle informazioni fornite dall’utente, accedendo alla schermata di visualizzazione dei prestiti attivi (vedere \hyperlink{UI-4}{UI-4}).
                \\Una volta identificato il prestito corrispondente, l’amministratore lo selezionerà e procederà con la registrazione dell’avvenuta restituzione (definita in \hyperlink{IF-13}{IF-13}). 
            \end{itemize}
        \end{tcolorbox}
        
        \begin{tcolorbox}[breakable, colback=white,colframe=black!80!white,title=\textbf{Esigenze di Dati e Informazioni - DF}]
            \begin{itemize}[itemsep=6pt]
                \hypertarget{}{\item[\textbf{DF-1}]}\textbf{Archiviazione su file}
                \\L’intero archivio della biblioteca dovrà essere memorizzato, in maniera automatica, all’interno di una serie di file.\vspace{.2cm}
                \\I dati saranno suddivisi e memorizzati in tre file distinti: 
                \begin{itemize}[label=$\bullet$]
                    \item Un file per la memorizzazione dell’elenco dei libri;
                    \item Un file per la memorizzazione dell’elenco degli utenti;
                    \item Un file per la memorizzazione dell’elenco dei prestiti. 
                \end{itemize}
                L’archiviazione dovrà avvenire ogni qualvolta il software viene chiuso.
                \\Al riavvio dell’applicazione, il sistema dovrà eseguire automaticamente la lettura e il caricamento in memoria del contenuto di tutti e tre i file.      
                
                \hypertarget{DF-2}{\item[\textbf{DF-2}]}\textbf{Informazioni di un libro}
                \\Il sistema deve memorizzare, nel formato appropriato, tutte le informazioni relative ai libri presenti nella lista dei libri della biblioteca.\vspace{.2cm}
                \\I dati che dovranno essere memorizzati sono: 
                \begin{itemize}[label=$\bullet$]
                    \item Titolo;
                    \item Lista degli autori (nome e cognome); 
                    \item Anno di pubblicazione; 
                    \item Codice identificativo (univoco); 
                    \item Numero di copie disponibili. 
                \end{itemize}
                Nello specifico: 
                \begin{itemize}[label=$\bullet$]
                    \item Il campo “Titolo” dovrà essere una stringa, con una lunghezza massima di cento caratteri;
                    \item I campi “Nome” e “Cognome” di ogni autore dovranno essere stringhe di caratteri alfabetici (con una lunghezza massima di venticinque caratteri);
                    \item Il campo “Anno di pubblicazione” dovrà essere un numero a quattro cifre, compreso tra zero e l’anno corrente;
                    \item Il campo “Codice identificativo univoco” dovrà essere il codice ISBN del libro, e dunque conforme allo standard ISO 2108: 2017;
                    \item Il campo “Numero di copie disponibili” dovrà essere un numero compreso tra zero e cento (definita come disponibilità massima per un singolo volume). 
                \end{itemize}
                
                \hypertarget{DF-3}{\item[\textbf{DF-3}]}\textbf{Informazioni dell'utente}
                \\Il sistema deve memorizzare, nel formato appropriato, tutte le informazioni relative agli utenti della biblioteca.\vspace{.2cm}
                I dati che dovranno essere memorizzati sono: 
                \begin{itemize}[label=$\bullet$]
                    \item Nome;
                    \item Cognome;
                    \item Matricola (univoca);
                    \item E-mail istituzionale; 
                    \item Lista dei libri attualmente in prestito e data prevista per la restituzione (di ciascun libro).
                \end{itemize}
                Nello specifico: 
                \begin{itemize}[label=$\bullet$]
                    \item I campi “Nome” e “Cognome” dovranno essere stringhe di caratteri alfabetici (con una lunghezza massima di venticinque caratteri);
                    \item Il campo “Matricola” dovrà essere una stringa di dieci cifre numeriche;
                    \item Il campo “E-mail” dovrà aderire allo standard RFC 5322, e come campo “dominio” dovrà avere “studenti.unisa.it”. 
                    
                    \hypertarget{DF-3.1}{\item[\textbf{DF-3.1}]}\textbf{Numero di telefono dell'utente}
                    \\Il sistema offre la possibilità di estendere l’insieme dei campi memorizzati per ogni utente (definiti in \hyperlink{DF-3}{DF-3}) includendo un’ulteriore informazione: 
                    \begin{itemize}[label=$\bullet$]
                        \item Numero di telefono.
                    \end{itemize}
                    Quest'ultimo dovrà essere conforme allo standard ITU E.164, in particolar modo al formato italiano (che prevede “39” come codice paese, ed una lunghezza pari a dieci cifre). 
                \end{itemize}

                \hypertarget{DF-4}{\item[\textbf{DF-4}]}\textbf{Salvataggio delle informazioni di un libro}
                \\Il sistema deve memorizzare, nel formato appropriato, tutte le informazioni relative ai prestiti effettuati agli utenti della biblioteca.\vspace{.2cm}
                \\I dati che dovranno essere memorizzati sono: 
                \begin{itemize}[label=$\bullet$]
                    \item Identificativo (univoco) dell’utente che ha richiesto il prestito;
                    \item Identificativo (univoco) del libro prestato; 
                    \item Data di avvenuto (inizio) prestito; 
                    \item Data prevista per la restituzione;
                    \item Data effettiva di restituzione (solo per i prestiti completati). 
                \end{itemize}
                Nello specifico: 
                \begin{itemize}[label=$\bullet$]
                    \item L’identificativo dell’utente corrisponderà alla propria matricola (vedere \hyperlink{DF-2}{DF-2} per le specifiche sul formato); 
                    \item L’identificativo del libro corrisponderà al suo codice ISBN (vedere \hyperlink{DF-3}{DF-3} per le specifiche sul formato);
                    \item I campi relativi alle date dovranno avere il seguente formato: “AAAA-MM-GG”. 
                \end{itemize}
            \end{itemize}
        \end{tcolorbox}
        
        \begin{tcolorbox}[breakable, colback=white,colframe=black!80!white,title=\textbf{Interfaccia Utente - UI}]
            \begin{itemize}[itemsep=6pt]
                \hypertarget{UI-1}{\item[\textbf{UI-1}]}\textbf{Interfaccia di tipo grafico}
                \\L’amministratore potrà interagire con l’applicazione mediante un’interfaccia di tipo grafico, evitando l’uso diretto di interfacce testuali o della linea di comando.\vspace{.2cm}
                \\L'uso di un'interfaccia consente, mediante un insieme di schermate di visualizzazione sequenziali (vedere \hyperlink{UI-2}{UI-2}, \hyperlink{UI-3}{UI-3} e \hyperlink{UI-4}{UI-4} per la loro descrizione dettagliata), la navigazione tra le varie funzionalità messe a disposizione dal sistema. 
                
                \hypertarget{UI-2}{\item[\textbf{UI-2}]}\textbf{Visualizzazione lista dei libri}
                \\L'applicazione dovrà includere una schermata dedicata alla visualizzazione e alla gestione dell’elenco completo dei libri presenti nella lista dei libri.
                \\Di default, all’apertura dell’applicazione verrà mostrata questa schermata.\vspace{.2cm}
                \\L'elenco dei libri verrà presentato all’amministratore in formato tabellare, rispettando l’ordinamento definito in \hyperlink{IF-4}{IF-4}.
                \\La schermata supporterà la paginazione, limitando la visualizzazione ad un massimo di venti libri per pagina.
                \\Saranno previste le seguenti funzionalità di interazione: 
                \begin{enumerate}
                    \item Barra di ricerca: sarà disponibile un campo di testo dedicato per effettuare la ricerca testuale (descritta in \hyperlink{IF-5}{IF-5}); 
                    \item Filtri avanzati (vedere \hyperlink{IF-14}{IF-14}): l’amministratore avrà modo di aprire, premendo il pulsante “Filtri”, una schermata apposita dove sarà possibile selezionare uno o più criteri, in base ai quali verrà limitata la visualizzazione dell’elenco;
                    \item Pulsanti operativi per l’esecuzione delle operazioni principali:
                    \begin{itemize}[label=$\bullet$]
                        \item Pulsante per accedere alla visualizzazione delle informazioni complete relative ai singoli libri (vedere \hyperlink{DF-2}{DF-2});
                        \item Pulsante “Aggiungi” per l’inserimento di un nuovo libro all’interno dell’elenco (vedere \hyperlink{IF-1}{IF-1}); 
                        \item Pulsante “Modifica” per la modifica delle informazioni relative al libro selezionato (vedere \hyperlink{IF-2}{IF-2});
                        \item Pulsante “Rimuovi” per la rimozione di un libro dall’elenco (vedere \hyperlink{IF-3}{IF-3}). 
                    \end{itemize} 
                \end{enumerate}
                Tutte le operazioni di rimozione dovranno essere precedute da una schermata di conferma, che richiede l’esplicito consenso dall’amministratore (ottenuto mediante la pressione del pulsante “Sì”) prima dell’esecuzione.
                \\Inoltre, tutte le operazioni di gestione della lista dei libri, nonché le stesse schermate di conferma, dovranno includere un pulsante di annullamento che permette di tornare alla schermata precedente. 
                
                \hypertarget{UI-3}{\item[\textbf{UI-3}]}\textbf{Visualizzazione lista degli utenti}
                \\L'applicazione dovrà includere una schermata dedicata alla visualizzazione e alla gestione dell’elenco completo degli utenti presenti nella lista degli utenti.\vspace{.2cm}
                \\L'elenco degli utenti verrà presentato all’amministratore in formato tabellare, rispettando l’ordinamento definito in \hyperlink{IF-9}{IF-9}.
                \\La schermata supporterà la paginazione, limitando la visualizzazione ad un massimo di venti utenti per pagina.
                \\Saranno previste le seguenti funzionalità di interazione: 
                \begin{enumerate}
                    \item Barra di ricerca: sarà disponibile un campo di testo dedicato per effettuare la ricerca testuale (descritta in \hyperlink{IF-10}{IF-10});
                    \item Pulsanti operativi per l’esecuzione delle operazioni principali:
                    \begin{itemize}[label=$\bullet$]
                        \item Pulsante per accedere alla visualizzazione delle informazioni complete relative ai singoli utenti (vedere \hyperlink{DF-3}{DF-3}); 
                        \item Pulsante “Aggiungi” per l’inserimento di un nuovo utente all’interno dell’elenco (vedere \hyperlink{IF-6}{IF-6});
                        \item Pulsante “Modifica” per la modifica delle informazioni relative all’utente selezionato (vedere \hyperlink{IF-7}{IF-7}); 
                        \item Pulsante “Rimuovi” per la rimozione di un utente dall’elenco (vedere \hyperlink{IF-8}{IF-8}).
                    \end{itemize}
                \end{enumerate}
                Tutte le operazioni di rimozione dovranno essere precedute da una schermata di conferma, che richiede l’esplicito consenso dall’amministratore (ottenuto mediante la pressione del pulsante “Sì”) prima dell’esecuzione.
                \\Inoltre, tutte le operazioni di gestione della lista degli utenti, nonché le stesse schermate di conferma, dovranno includere un pulsante di annullamento che permette di tornare alla schermata precedente.  
                
                \hypertarget{UI-4}{\item[\textbf{UI-4}]}\textbf{Visualizzazione lista dei prestiti}
                \\L'applicazione dovrà includere una schermata dedicata alla visualizzazione e alla gestione dell’elenco completo dei prestiti.\vspace{.2cm}
                \\L'elenco dei prestiti verrà presentato all’utente in formato tabellare, rispettando l’ordinamento definito in \hyperlink{IF-12}{IF-12}.
                \\La schermata supporterà la paginazione, limitando la visualizzazione ad un massimo di venti prestiti per pagina.
                \\Saranno previste le seguenti funzionalità di interazione: 
                \begin{enumerate}
                    \item Messa in evidenza dei ritardi: nel caso dei prestiti che presentano un ritardo, la data prevista per la restituzione verrà evidenziata in rosso, in modo tale da segnalare visivamente il ritardo all’amministratore.
                    \\Ad ogni prestito in ritardo sarà inoltre associata una casella, da spuntare in caso di avvenuta telefonata all’utente da parte dell’amministratore (vedere \hyperlink{BF-1}{BF-1}).
                    \item Pulsanti operativi per l’esecuzione delle operazioni principali:
                    \begin{itemize}[label=$\bullet$]
                        \item Pulsante “Prestiti” per la visualizzazione delle informazioni complete relative ai singoli prestiti (vedere \hyperlink{DF-4}{DF-4}); 
                        \item Pulsante “Inserisci” per la registrazione di un nuovo prestito (vedere \hyperlink{IF-11}{IF-11});
                        \item Pulsante “Estendi” per l’estensione di un prestito già esistente (vedere \hyperlink{IF-15}{IF-15}); 
                        \item Pulsante “Restituzione” per la registrazione di una restituzione (vedere \hyperlink{IF-13}{IF-13}). 
                    \end{itemize}
                \end{enumerate}
                Tutte le operazioni di registrazione dovranno essere precedute da una schermata di conferma, che richiede l’esplicito consenso dall’amministratore (ottenuto mediante la pressione del pulsante “Sì”) prima dell’esecuzione.
                \\Inoltre, tutte le operazioni di gestione della lista dei prestiti, nonché le stesse schermate di conferma, dovranno includere un pulsante di annullamento che permette di tornare alla schermata precedente.  
            \end{itemize}
        \end{tcolorbox}

    \newpage
    \subsection{Requisiti Non Funzionali}
        \begin{tcolorbox}[breakable, colback=white,colframe=black!80!white,title=\textbf{Requisiti Non Funzionali - FC}]
            \begin{itemize}[itemsep=6pt]
                \item[\textbf{FC-1}]\textbf{Affidabilità dei dati}
                \\Lo stato corrente dell'archivio (vedere \hyperlink{DF-2}{DF-2}, \hyperlink{DF-3}{DF-3} e \hyperlink{DF-4}{DF-4} per le specifiche sui dati memorizzati) dovrà essere salvato immediatamente prima della chiusura dell’applicazione, in modo che i dati possano essere protetti da perdite, corruzioni e modifiche non autorizzate.
                \\I dati dovranno essere sempre recuperabili. 
                
                \item[\textbf{FC-2}]\textbf{Intuitività e semplicità dell'interfaccia}
                \\L’interfaccia \hyperlink{UI-1}{UI-1} dovrà essere progettata col fine di massimizzare l’intuitività e la semplicità operativa.\vspace{.2cm}
                \\I due criteri utilizzati per la misurazione di tali aspetti sono: 
                \begin{enumerate}
                    \item Efficienza nell’eseguibilità: l’esecuzione delle operazioni principali sarà efficiente, in modo tale che il completamento di queste ultime non richieda più di quattro pressioni di pulsanti;
                    \item Coerenza del layout: Il layout dovrà essere chiaro, privo di eccessivo affollamento e ricco di elementi autoesplicativi, al fine di rendere la navigazione prevedibile e facilmente apprendibili. 
                \end{enumerate}
                Inoltre, i pulsanti di navigazione principali e i pulsanti di azione manterranno un posizionamento semanticamente coerente rispetto alla loro funzionalità, attraverso tutte le schermate operative.  
                
                \item[\textbf{FC-3}]\textbf{Cifratura dati sensibili}
                \\Il sistema garantisce la riservatezza dei dati sensibili degli utenti (specificati in \hyperlink{DF-3}{DF-3}), offuscandoli mediante un algoritmo crittografico prima della loro memorizzazione su file (come definito in \hyperlink{DF-1}{DF-1}).\vspace{.2cm}
                \\Nello specifico, l’algoritmo di cifratura utilizzato sarà il Cifrario di Cesare (algoritmo di sostituzione monoalfabetica).
                \\Si garantisce la totale reversibilità del processo di cifratura.  
            \end{itemize}
        \end{tcolorbox}

    \newpage
    \subsection{Diagramma UML dei Casi d'Uso}
        
            \begin{adjustbox}{width=.8\paperwidth, height=.8\paperheight, center}
                \includesvg{usecases.svg}    
            \end{adjustbox}
            \captionof{figure}{Diagramma UML dei Casi d'Uso}
  
        
    \newpage
    \subsection{Descrizione dei Casi d'Uso}
        
\end{document}

