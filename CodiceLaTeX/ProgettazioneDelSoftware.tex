\documentclass[12pt, a4paper]{article}

\usepackage[top=2.3cm, bottom=2cm, left=2cm, right=2cm, headheight=15pt]{geometry}
\usepackage{graphicx}
\usepackage{svg}
\usepackage[italian]{babel}
\usepackage{lastpage}
\usepackage{adjustbox}
\usepackage{enumitem}
\usepackage[most]{tcolorbox}
%\usepackage{xcolor}
\usepackage[colorlinks=true]{hyperref}
\usepackage{fancyhdr}
\pagestyle{fancy}
\usepackage{caption}
\renewcommand{\contentsname}{Indice}
\renewcommand{\sectionmark}[1]{\markboth{#1}{#1}}


\begin{document}
    \begin{titlepage}
        \fancyhf{}
		\centering
		
		{\Large \textsc{Ingegneria del Software - A.A. 2025/2026}}\\[.4cm]
        {\Large Applicazione per la gestione di una Biblioteca universitaria}\\[4cm]
		
		{\huge \textbf{Progettazione del Software}}\\[0.5cm]
		
		\large { 
			\textit{
            Luisa Genovese\\
            Erica Brancaccio\\
            Paolo Alfé\\
            Francesco Altieri}
		}
		
		\vspace{14cm}
		
		{\large \today}
		
		
	\end{titlepage}

    \fancyhf{}
    \fancyhead[R]{pagina \thepage\ di \pageref*{LastPage}}
    \pagestyle{fancy}
    \hypersetup{linkcolor=black}
    \begingroup
        \let\bfseries\normalfont
        \tableofcontents
    \endgroup
    \newpage

    \fancyhead[L]{\nouppercase{\leftmark}}
    \section{Progettazione del Software}
    \subsection{Decomposizione in moduli}
        \subsubsection{Descrizione delle classi individuate}
            Il diagramma delle classi rappresenta il sistema della biblioteca, con le funzionalità di gestione (ovvero di aggiunta, modifica e rimozione dall'archivio) dei libri, degli utenti e dei prestiti, nonché quella di lettura/salvataggio di dati all'atto dell'apertura/chiusura dell'applicazione.
            \begin{itemize}[label=$\bullet$]
                \setlength\itemsep{0.1em}
                \item \textbf{Biblioteca}: classe centrale dell'applicazione, che modella una biblioteca. Mantiene e gestisce le tre collezioni principali mediante gli attributi \texttt{libri}, \texttt{utenti} e \texttt{prestiti}. Fornisce i metodi per la serializzazione e deserializzazione (salvataggio e lettura) dei dati su file;
                \item \textbf{Maschera}: classe che modella una maschera, utilizzata per la verifica dei campi di un oggetto. Possiede gli attributi \texttt{lunghezza} e \texttt{maschera}, e metodi per la gestione di questi ultimi (getter e setter); 
                \item \textbf{Prestiti}: classe che modella un insieme di prestiti. Possiede l'attributo \texttt{prestiti} (una collezione di oggetti di tipo Prestito). Implementa l'interfaccia Archiviabile, ereditando i metodi fondamentali per la manipolazione della collezione. Fornisce un metodo per la visualizzazione dei prestiti (getter) e un metodo di filtraggio per visualizzare solo i prestiti che rispettano il criterio applicato; 
                \item \textbf{Utenti}: classe che modella un insieme di utenti. Possiede gli attributi \texttt{chiaviMatricole} e \texttt{utenti} (due collezioni di oggetti di tipo Utente). Implementa le interfacce Archiviabile e Mappabile, ereditando i metodi fondamentali per la manipolazione delle collezioni. Fornisce un metodo per la visualizzazione degli utenti (getter) e un metodo per effettuare una ricerca all’interno della collezione;
                \item \textbf{Libri}: classe che modella un insieme di libri. Possiede gli attributi \texttt{chiaviISBN} e \texttt{libri} (due collezioni di oggetti di tipo Libro). Implementa le interfacce Archiviabile e Mappabile, ereditando i metodi fondamentali per la manipolazione delle collezioni. Fornisce un metodo per la visualizzazione dei libri (getter) e un metodo per effettuare una ricerca all’interno della collezione;
                \item \textbf{Archiviabile}: interfaccia parametrizzata implementata dalle classi Prestiti, Utenti e Libri. Definisce i tre metodi per le operazioni fondamentali di aggiunta, modifica e rimozione dell'elemento passato come parametro; 
                \item \textbf{Mappabile}: interfaccia parametrizzata implementata dalle classi Utenti e Libri. Definisce i metodi per l'accesso e la ricerca di un valore attraverso una chiave;
                \item \textbf{Filtro}: classe enumerativa che definisce tre stati di filtraggio: \texttt{TUTTI}, \texttt{ATTIVI} e \texttt{CONCLUSI}. Questa classe è utilizzata per l'applicazione di criteri di selezione sull'insieme dei prestiti; 
                \item \textbf{Prestito}: classe che modella un prestito. I suoi attributi sono: \texttt{matricolaUtente}, \texttt{codiceISBNLibro}, \texttt{dataInizio}, \texttt{dataScadenza} e \texttt{dataRestituzione}. Fornisce vari metodi per la modifica e visualizzazione di questi ultimi (getter e setter) e per la verifica della correttezza del formato dei dati; 
                \item \textbf{Utente}: classe che specializza Persona. Modella un utente del sistema. I suoi attributi sono: \texttt{matricolaUtente}, \texttt{email} e \texttt{prestitiAttivi} (una collezione di oggetti di tipo Prestito). Fornisce vari metodi per la modifica e visualizzazione di questi ultimi (getter e setter), per la gestione della collezione di prestiti e per la verifica della correttezza del formato dei dati; 
                \item \textbf{Libro}: classe che modella un libro. I suoi attributi sono: \texttt{titolo}, \texttt{autori} (una collezione di oggetti di tipo Autore), \texttt{annoPubblicazione}, \texttt{codiceISBNLibro} e \texttt{copieDisponibili}. Fornisce vari metodi per la modifica e visualizzazione di questi ultimi (getter e setter), per la gestione della collezione di autori e per la verifica della correttezza del formato dei dati;
                \item \textbf{Matricola}: classe che modella una matricola (ovvero il codice identificativo univoco) associata ad un utente. Possiede l'attributo \texttt{matricola}, il metodo per la sua visualizzazione (getter) e un metodo per verificare la correttezza del proprio formato; 
                \item \textbf{ISBN}: classe che modella un codice ISBN (ovvero il codice identificativo univoco) associato ad un libro. Possiede l'attributo \texttt{codiceISBN}, il metodo per la sua visualizzazione (getter) e un metodo per verificare la correttezza del proprio formato; 
                \item \textbf{Autore}: classe che specializza Persona. Modella l'entità autore associata ad un libro.
                \item \textbf{Persona}: classe astratta che funge da superclasse per Utente e Autore. Definisce gli attributi fondamentali di una persona, ovvero \texttt{nome} e \texttt{cognome}. Fornisce metodi per la loro gestione (getter e setter) e per la verifica della correttezza del loro formato.
            \end{itemize} 
            All'interno del diagramma viene omessa l’implementazione dell’interfaccia Serializable al fine di aumentare la propria leggibilità. Suddetta interfaccia dovrà essere implementata dalle classi Biblioteca, Prestiti, Libri, Utenti, Prestito, Libro, Utente, Autore, Matricola e ISBN.\vspace{0.2cm}
            \\Per lo stesso motivo, all'interno del diagramma vengono omesse le classi Controller, così come le relazioni tra queste ultime ed i modelli. 
        \newpage

        \subsubsection{Diagramma delle classi}
        \begin{figure}[h!]
            \begin{adjustbox}{width=.92\paperwidth, height=.75\paperheight, center, margin= 0cm .6cm 0cm .1cm}
                \includesvg[inkscapelatex=false]{classdiagram.svg}    
            \end{adjustbox}
            \captionof{figure}{Diagramma delle Classi} 
        \end{figure}
        \newpage

         \subsubsection{Diagramma dei package (essenziale)}
        \begin{figure}[h!]
            \begin{adjustbox}{width=.33\paperwidth, height=.5\paperheight, center, margin= 0cm .6cm 0cm .1cm}
                \includesvg[inkscapelatex=false]{diagrammaDeiPackageNonDettagliato.svg}    
            \end{adjustbox}
            \captionof{figure}{Diagramma dei Package relativo ai modelli} 
        \end{figure}
        \newpage
        
        \subsubsection{Diagramma dei package (completo)}
        \begin{figure}[h!]
            \begin{adjustbox}{width=.7\paperwidth, height=.75\paperheight, center, margin= 0cm .3cm 0cm .1cm}
                \includesvg[inkscapelatex=false]{diagrammaDeiPackage.svg}    
            \end{adjustbox}
            \captionof{figure}{Diagramma dei Package completo (incluse le classi Controller e View)} 
        \end{figure}
    \newpage   

    \subsection{Diagrammi di sequenza}
        Di seguito vengono presentati dei diagrammi di sequenza che descrivono i flussi esecutivi dei casi d’uso più rilevanti, utili per la comprensione delle funzionalità principali del sistema.
        \\Questi diagrammi descrivono l’interazione tra l’attore \emph{Amministratore} e gli oggetti del sistema coinvolti, ai fini di soddisfare i requisiti specificati. 
        \subsubsection{\textbf{C4: Inserimento di un libro}}
        Il seguente diagramma descrive l'esecuzione del caso d'uso C4.\vspace{.2cm}
        \\Per migliorare la leggibilità del diagramma, è stato omesso il flusso alternativo di annullamento dell’operazione tramite la pressione del pulsante “Annulla”.
        \\Per il flusso alternativo di inserimento errato o incompleto, è stata utilizzata la sintassi \texttt{loop-break}.
        \\Invece, per la descrizione del flusso alternativo relativo all'esistenza pregressa del libro all'interno dell'archivio, è stata utilizzata la sintassi \texttt{alt}.
        \begin{figure}[h!]
            \begin{adjustbox}{width=.84\paperwidth, height=.5\paperheight, center, margin= 0cm 0cm 0cm .1cm}
                \includesvg[inkscapelatex=false]{diagrammaC4InserimentoLibro.svg} 
            \end{adjustbox}
            \captionof{figure}{Diagramma di sequenza C4} 
        \end{figure}
        \newpage

        \subsubsection{\textbf{C5: Modifica di un libro}}
        Il seguente diagramma descrive l'esecuzione del caso d'uso C5.\vspace{.2cm}
        \\Per migliorare la leggibilità del diagramma, è stato omesso il flusso alternativo di annullamento dell’operazione tramite la pressione del pulsante “Annulla”.
        \\Per il flusso alternativo relativo ad una modifica errata o incompleta è stata utilizzata la sintassi \texttt{loop-break}.
        \begin{figure}[h!]
            \begin{adjustbox}{width=.84\paperwidth, height=.5\paperheight, center, margin= 0cm 0cm 0cm .1cm}
                \includesvg[inkscapelatex=false]{diagrammaC5ModificaLibro.svg} 
            \end{adjustbox}
            \captionof{figure}{Diagramma di sequenza C5} 
        \end{figure}
        \newpage

        \subsubsection{\textbf{C6: Rimozione di un libro}}
        Il seguente diagramma descrive l'esecuzione del caso d'uso C6.\vspace{.2cm}
        \\Per il flusso alternativo di annullamento dell’operazione tramite la pressione del pulsante “No” è stata utilizzata la sintassi \texttt{alt}.
        \begin{figure}[h!]
            \begin{adjustbox}{width=.84\paperwidth, height=.5\paperheight, center, margin= 0cm 0cm 0cm .1cm}
                \includesvg[inkscapelatex=false]{diagrammaC6RimuoviLibro.svg} 
            \end{adjustbox}
            \captionof{figure}{Diagramma di sequenza C6} 
        \end{figure}
        \newpage

        \subsubsection{\textbf{C7: Inserimento di un utente}}
        Il seguente diagramma descrive l'esecuzione del caso d'uso C7.\vspace{.2cm}
        \\Per migliorare la leggibilità del diagramma, è stato omesso il flusso alternativo di annullamento dell’operazione tramite la pressione del pulsante “Annulla”.
        \\Per il flusso alternativo di inserimento errato o incompleto, è stata utilizzata la sintassi \texttt{loop-break}.
        \\Invece, per la descrizione del flusso alternativo relativo all'esistenza pregressa dell'utente all'interno dell'archivio, è stata utilizzata la sintassi \texttt{alt}.
        \begin{figure}[h!]
            \begin{adjustbox}{width=.84\paperwidth, height=.5\paperheight, center, margin= 0cm 0cm 0cm .1cm}
                \includesvg[inkscapelatex=false]{diagrammaC7InserisciUtente.svg} 
            \end{adjustbox}
            \captionof{figure}{Diagramma di sequenza C7} 
        \end{figure}
        \newpage

        \subsubsection{\textbf{C8: Modifica di un utente}}
        Il seguente diagramma descrive l'esecuzione del caso d'uso C8.\vspace{.2cm}
        \\Per migliorare la leggibilità del diagramma, è stato omesso il flusso alternativo di annullamento dell’operazione tramite la pressione del pulsante “Annulla”.
        \\Per il flusso alternativo relativo ad una modifica errata o incompleta è stata utilizzata la sintassi \texttt{loop-break}.
        \begin{figure}[h!]
            \begin{adjustbox}{width=.84\paperwidth, height=.5\paperheight, center, margin= 0cm 0cm 0cm .1cm}
                \includesvg[inkscapelatex=false]{diagrammaC8ModificaUtente.svg} 
            \end{adjustbox}
            \captionof{figure}{Diagramma di sequenza C8} 
        \end{figure}
        \newpage

        \subsubsection{\textbf{C9: Rimozione di un utente}}
        Il seguente diagramma descrive l'esecuzione del caso d'uso C9.\vspace{.2cm}
        \\Per il flusso alternativo di annullamento dell’operazione tramite la pressione del pulsante “No” è stata utilizzata la sintassi \texttt{alt}.
        \begin{figure}[h!]
            \begin{adjustbox}{width=.84\paperwidth, height=.5\paperheight, center, margin= 0cm 0cm 0cm .1cm}
                \includesvg[inkscapelatex=false]{diagrammaC9RimuoviUtente.svg} 
            \end{adjustbox}
            \captionof{figure}{Diagramma di sequenza C9} 
        \end{figure}
        \newpage

        \subsubsection{\textbf{C10: Ricerca di un libro}}
        Il seguente diagramma descrive l'esecuzione del caso d'uso C10.\vspace{.2cm}
        \\All'interno del diagramma è stato scelto di utilizzare la sintassi \texttt{loop-break} per simulare la ricerca mediante l'inserimento di una sottostringa nella barra di ricerca.
        \\Ogni volta che viene aggiunto un carattere alla stringa, verrà effettuata una ricerca in tempo reale e verranno visualizzati tutti i libri i cui campi presenteranno almeno una corrispondenza con la sottostringa inserita.
        \\L'operazione terminerà quando l’amministratore selezionerà un libro.
        \begin{figure}[h!]
            \begin{adjustbox}{width=.84\paperwidth, height=.5\paperheight, center, margin= 0cm 0cm 0cm .1cm}
                \includesvg[inkscapelatex=false]{diagrammaC10RicercaLibro.svg} 
            \end{adjustbox}
            \captionof{figure}{Diagramma di sequenza C10} 
        \end{figure}
        \newpage

        \subsubsection{\textbf{C12: Ricerca di un utente}}
        Il seguente diagramma descrive l'esecuzione del caso d'uso C12.\vspace{.2cm}
        \\All'interno del diagramma è stato scelto di utilizzare la sintassi \texttt{loop-break} per simulare la ricerca mediante l'inserimento di una sottostringa nella barra di ricerca.
        \\Ogni volta che viene aggiunto un carattere alla stringa, verrà effettuata una ricerca in tempo reale e verranno visualizzati tutti gli utenti i cui campi presenteranno almeno una corrispondenza con la sottostringa inserita.
        \\L'operazione terminerà quando l’amministratore selezionerà un utente.
        \begin{figure}[h!]
            \begin{adjustbox}{width=.84\paperwidth, height=.5\paperheight, center, margin= 0cm 0cm 0cm .1cm}
                \includesvg[inkscapelatex=false]{diagrammaC12RicercaUtente.svg} 
            \end{adjustbox}
            \captionof{figure}{Diagramma di sequenza C12} 
        \end{figure}
        \newpage

        \subsubsection{\textbf{C13: Registrazione di un prestito}}
        Il seguente diagramma descrive l'esecuzione del caso d'uso C13.\vspace{.2cm}
        \\Per migliorare la leggibilità del diagramma, è stato omesso il flusso alternativo di annullamento dell’operazione tramite la pressione del pulsante “Annulla”.
        \\Per il flusso alternativo di inserimento errato o incompleto è stata utilizzata la sintassi \texttt{loop-break}.
        \\Invece, per la descrizione del flusso alternativo relativo all'inesistenza dell'utente e del libro all'interno dell'archivio, è stata utilizzata la sintassi \texttt{alt}.
        \begin{figure}[h!]
            \begin{adjustbox}{width=.84\paperwidth, height=.5\paperheight, center, margin= 0cm 0cm 0cm .1cm}
                \includesvg[inkscapelatex=false]{diagrammaC13RegistraPrestito.svg} 
            \end{adjustbox}
            \captionof{figure}{Diagramma di sequenza C13} 
        \end{figure}
        \newpage

        \subsubsection{\textbf{C14: Registrazione di una restituzione}}
        Il seguente diagramma descrive l'esecuzione del caso d'uso C14.\vspace{.2cm}
        \\Per il flusso alternativo di annullamento dell’operazione tramite la pressione del pulsante “No” è stata utilizzata la sintassi \texttt{alt}.
        \begin{figure}[h!]
            \begin{adjustbox}{width=.84\paperwidth, height=.5\paperheight, center, margin= 0cm 0cm 0cm .1cm}
                \includesvg[inkscapelatex=false]{diagrammaC14RegistraRestituzione.svg} 
            \end{adjustbox}
            \captionof{figure}{Diagramma di sequenza C14} 
        \end{figure}
        \newpage

        \subsubsection{\textbf{C15: Estensione di un prestito}}
        Il seguente diagramma descrive l'esecuzione del caso d'uso C15.\vspace{.2cm}
        \\Per migliorare la leggibilità del diagramma, è stato omesso il flusso alternativo di annullamento dell’operazione tramite la pressione del pulsante “Annulla”.
        \\Per il flusso alternativo relativo ad una modifica errata o incompleta è stata utilizzata la sintassi \texttt{loop-break}. 
        \begin{figure}[h!]
            \begin{adjustbox}{width=.84\paperwidth, height=.5\paperheight, center, margin= 0cm 0cm 0cm .1cm}
                \includesvg[inkscapelatex=false]{diagrammaC15EstendiPrestito.svg} 
            \end{adjustbox}
            \captionof{figure}{Diagramma di sequenza C15} 
        \end{figure}
    \newpage
    
    \subsection{Principi di buona progettazione}
        La progettazione di dettaglio dell'applicazione è stata portata a termine tenendo a mente i principi di buona progettazione, garantendo un'alta ortogonalità, manutenibilità, modularità e chiarezza.\vspace{.2cm}
        \\Di seguito riportiamo le tabelle con i livelli di coesione e di accoppiamento individuati tra i moduli del nostro sistema: 
        \subsubsection{Livelli di Coesione}
            \begin{adjustbox}{width=.7\paperwidth, height=.66\paperheight, center, margin= 0cm 0cm 0cm .1cm}
                \includegraphics{coesione.png}    
            \end{adjustbox}
            \captionof{figure}{Tabella dei livelli di coesione dei moduli} 
        \newpage

        \subsubsection{Livelli di Accoppiamento}
            \begin{adjustbox}{width=.8\paperwidth, height=.8\paperheight, center, margin= 0cm 0cm 0cm .1cm}
                \includegraphics{accoppiamento.png}    
            \end{adjustbox}
            \captionof{figure}{Tabella dei livelli di accoppiamento tra i moduli} 
        \newpage

        Tutte le classi sono state progettate in accordo con il principio di \emph{Separazione delle preoccupazioni}: ogni modulo contiene solo ed esclusivamente attributi e metodi che hanno a che fare con la propria funzionalità.
        \\Ciò garantisce di ottenere, nella maggior parte dei casi, il più alto livello di coesione possibile (\emph{coesione funzionale}).\vspace{.2cm}
        \\Fin dalle prime fasi della progettazione, vista la natura stessa dell'applicazione è emerso un insieme di funzionalità comuni alle principali collezioni di dati (Prestiti, Libri ed Utenti).
        \\Per evitare ridondanze, e rispettare quanto affermato dal principio di progettazione \emph{DRY} (\emph{Don't Repeat Yourself}), tali funzionalità sono state astratte ed inserite nelle interfacce \emph{Archiviabile} e \emph{Mappabile}.
        \\Le classi che implementano tali interfacce si preoccuperanno di ridefinire il comportamento specifico dei metodi ereditati da esse, mentre la loro funzionalità di base resterà sempre la stessa.
        \\In questo modo è possibile ottenere un codice chiaro (secondo quanto affermato dal principio \emph{KISS} - \emph{Keep It Simple}) e facilmente manutenibile.\vspace{.2cm}

        \subsubsection{Analisi della scalabilità}
            Per aumentare la scalabilità del sistema, sarebbe stato possibile rendere la classe \emph{Utente} astratta, e successivamente specializzarla nelle diverse tipologie di utenti contemplati dall'applicazione.
            \\È stato scelto di non adottare questa strategia per due motivi:
            \begin{enumerate}
                \item Un'ulteriore relazione di specializzazione avrebbe comportato un alto livello di accoppiamento;
                \item Attualmente, il nostro sistema è in grado di gestire un'unica tipologia di utenti, ovvero gli studenti, quindi sarebbe stata un'aggiunta pressoché inutile (è stato seguito il principio \emph{YAGNI} - \emph{You Aren't Going to Need It}).
            \end{enumerate}
        
        \subsubsection{Strategie per la riduzione dell'Accoppiamento}
            Utilizzando ampliamente meccanismi come l'\emph{incapsulamento} e l'\emph{astrazione} mediante l'introduzione di interfacce, è stato possibile ridurre al minimo possibile l'accoppiamento tra i moduli, nonché aumentare la robustezza generale del sistema.\vspace{.2cm}
            \\Uno degli aspetti su cui sarebbe possibile effettuare delle ottimizzazioni è l'uso dell'\emph{ereditarietà}: per questioni semantiche (e dettate dalla natura del linguaggio di programmazione utilizzato, ovvero \emph{Java}), la struttura del nostro sistema prevede due relazioni di specializzazione (\emph{Utente} $\rightarrow$ \emph{Persona} e \emph{Autore} $\rightarrow$ \emph{Persona}), che portano con sé un livello di accoppiamento estremamente elevato (\emph{accoppiamento per contenuti}).\vspace{.2cm}
            \\Una possibile soluzione sarebbe quella di predilire l'\emph{associazione} all'\emph{ereditarietà}: anziché far dipendere l'implementazione di \emph{Utente} interamente da quella di \emph{Persona}, si potrebbe pensare di far mantenere a \emph{Utente} un riferimento ad una classe avente come attributi \emph{nome} e \emph{cognome}, riducendo di molto l'accoppiamento.
        
\end{document}
