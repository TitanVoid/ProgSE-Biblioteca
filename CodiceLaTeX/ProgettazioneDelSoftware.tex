\documentclass[12pt, a4paper]{article}

\usepackage[top=2.3cm, bottom=2cm, left=2cm, right=2cm, headheight=15pt]{geometry}
\usepackage{graphicx}
\usepackage{svg}
\usepackage[italian]{babel}
\usepackage{lastpage}
\usepackage{adjustbox}
\usepackage{enumitem}
\usepackage[most]{tcolorbox}
%\usepackage{xcolor}
\usepackage[colorlinks=true]{hyperref}
\usepackage{fancyhdr}
\pagestyle{fancy}
\usepackage{caption}
\renewcommand{\contentsname}{Indice}
\renewcommand{\sectionmark}[1]{\markboth{#1}{#1}}


\begin{document}
    \begin{titlepage}
        \fancyhf{}
		\centering
		
		{\Large \textsc{Ingegneria del Software - A.A. 2025/2026}}\\[.4cm]
        {\Large Applicazione per la gestione di una Biblioteca universitaria}\\[4cm]
		
		{\huge \textbf{Progettazione del Software}}\\[0.5cm]
		
		\large { 
			\textit{
            Luisa Genovese\\
            Erica Brancaccio\\
            Paolo Alfé\\
            Francesco Altieri}
		}
		
		\vspace{14cm}
		
		{\large \today}
		
		
	\end{titlepage}

    \fancyhf{}
    \fancyhead[R]{pagina \thepage\ di \pageref*{LastPage}}
    \pagestyle{fancy}
    \hypersetup{linkcolor=black}
    \begingroup
        \let\bfseries\normalfont
        \tableofcontents
    \endgroup
    \newpage

    \fancyhead[L]{\nouppercase{\leftmark}}
    \section{Progettazione del Software}
    \subsection{Decomposizione in moduli}
        \subsubsection{Descrizione delle classi individuate}
            Il diagramma delle classi rappresenta il sistema della biblioteca, con le funzionalità di gestione (ovvero di aggiunta, modifica e rimozione dall'archivio) dei libri, degli utenti e dei prestiti, nonché quella di lettura/salvataggio di dati all'atto dell'apertura/chiusura dell'applicazione.
            \begin{itemize}[label=$\bullet$]
                \setlength\itemsep{0.1em}
                \item \textbf{Biblioteca:} classe centrale dell'applicazione, che modella una biblioteca. Mantiene e gestisce le tre collezioni principali mediante gli attributi “libri”, “utenti” e “prestiti”. Fornisce i metodi per la serializzazione e deserializzazione (salvataggio e lettura) dei dati su file;
                \item \textbf{Maschera:} classe che modella una maschera, utilizzata per la verifica dei campi di un oggetto. Possiede gli attributi “lunghezza” e “maschera”, e metodi per la gestione di questi ultimi (getter e setter); 
                \item \textbf{Prestiti:} classe che modella un insieme di prestiti. Possiede l'attributo "prestiti" (una collezione di oggetti di tipo Prestito). Implementa l'interfaccia Archiviabile, ereditando i metodi fondamentali per la manipolazione della collezione. Fornisce un metodo per la visualizzazione dei prestiti (getter) e un metodo di filtraggio per visualizzare solo i prestiti che rispettano il criterio applicato; 
                \item \textbf{Utenti:} classe che modella un insieme di utenti. Possiede gli attributi “chiaviMatricole” e “utenti” (due collezioni di oggetti di tipo Utente). Implementa le interfacce Archiviabile e Mappabile, ereditando i metodi fondamentali per la manipolazione delle collezioni. Fornisce un metodo per la visualizzazione degli utenti (getter) e un metodo per effettuare una ricerca all’interno della collezione;
                \item \textbf{Libri:} classe che modella un insieme di libri. Possiede gli attributi “chiaviISBN” e “libri” (due collezioni di oggetti di tipo Libro). Implementa le interfacce Archiviabile e Mappabile, ereditando i metodi fondamentali per la manipolazione delle collezioni. Fornisce un metodo per la visualizzazione dei libri (getter) e un metodo per effettuare una ricerca all’interno della collezione;
                \item \textbf{Archiviabile:} interfaccia parametrizzata implementata dalle classi Prestiti, Utenti e Libri. Definisce i tre metodi per le operazioni findamentali di aggiunta, modifica e rimozione dell'elemento passato come parametro; 
                \item \textbf{Mappabile:} interfaccia parametrizzata implementata dalle classi Utenti e Libri. Definisce i metodi per l'accesso e la ricerca di un valore attraverso una chiave;
                \item \textbf{Filtro:} classe enumerativa che definisce tre stati di filtraggio: TUTTI, ATTIVI e CONCLUSI. Questa classe è utilizzata per l'applicazione di criteri di selezione sull'insieme dei prestiti; 
                \item \textbf{Prestito:} classe che modella un prestito. I suoi attributi sono: “matricolaUtente”, “codiceISBNLibro”, “dataInizio”, “dataScadenza” e “dataRestituzione”. Fornisce vari metodi per la modifica e visualizzazione di questi ultimi (getter e setter) e per la verifica della correttezza del formato dei dati; 
                \item \textbf{Utente:} classe che specializza Persona. Modella un utente del sistema. I suoi attributi sono: “matricolaUtente”, “email” e “prestitiAttivi” (una collezione di oggetti di tipo Prestito). Fornisce vari metodi per la modifica e visualizzazione di questi ultimi (getter e setter), per la gestione della collezione di prestiti e per la verifica della correttezza del formato dei dati; 
                \item \textbf{Libro:} classe che modella un libro. I suoi attributi sono: “titolo”, “autori” (una collezione di oggetti di tipo Autore), “annoPubblicazione”, “codiceISBNLibro” e “copieDisponibili”. Fornisce vari metodi per la modifica e visualizzazione di questi ultimi (getter e setter), per la gestione della collezione di autori e per la verifica della correttezza del formato dei dati;
                \item \textbf{Matricola:} classe che modella una matricola (ovvero il codice identificativo univoco) associata ad un utente. possiede l'attributo “matricola”, il metodo per la sua visualizzazione (getter) e un metodo per verificare la correttezza del proprio formato; 
                \item \textbf{ISBN:} classe che modella un codice ISBN (ovvero il codice identificativo univoco) associato ad un libro. Possiede l'attributo “codiceISBN”, il metodo per la sua visualizzazione (getter) e un metodo per verificare la correttezza del proprio formato; 
                \item \textbf{Autore:} classe che specializza Persona. Modella l'entità autore associata ad un libro.
                \item \textbf{Persona:} classe astratta che funge da superclasse per Utente e Autore. Definisce gli attributi fondamentali di una persona, ovvero “nome” e “cognome”. Fornisce metodi per la loro gestione (getter e setter) e per la verifica della correttezza del loro formato.
            \end{itemize} 
            All'interno del diagramma viene omessa l’implementazione dell’interfaccia Serializable al fine di aumentare la propria leggibilità. Suddetta interfaccia dovrà essere implementata dalle classi Biblioteca, Prestiti, Libri, Utenti, Prestito, Libro, Utente, Autore, Matricola e ISBN.\vspace{0.2cm}
            \\Per lo stesso motivo, all'interno del diagramma vengono omesse le classi Controller e le relazioni tra queste ultime ed i modelli. 
        \newpage

        \subsubsection{Diagramma delle classi}
        \begin{figure}[h!]
            \begin{adjustbox}{width=.92\paperwidth, height=.75\paperheight, center, margin= 0cm .6cm 0cm .1cm}
                \includesvg{classdiagram.svg}    
            \end{adjustbox}
            \captionof{figure}{Diagramma delle Classi} 
        \end{figure}
    \newpage

    \subsection{Diagrammi di sequenza}
        Di seguito vengono presentati dei diagrammi di sequenza che descrivono i flussi esecutivi dei casi d’uso più rilevanti, utili per la comprensione delle funzionalità principali del sistema.
        \\Questi diagrammi descrivono l’interazione tra l’attore \emph{Amministratore} e gli oggetti del sistema coinvolti, ai fini di soddisfare i requisiti specificati. 
        \subsubsection{\textbf{C4: Inserimento di un libro}}
        Il seguente diagramma descrive l'esecuzione del caso d'uso C4.\vspace{.2cm}
        \\Per migliorare la leggibilità del diagramma, abbiamo omesso il flusso alternativo di annullamento dell’operazione tramite il pulsante “Annulla”. Per il flusso alternativo di inserimento errato o incompleto abbiamo usato la sintassi del loop-break. Per quanto riguarda la sintassi per la descrizione del flusso alternativo dell’univocità del libro, abbiamo usato un alt.
        \begin{figure}[h!]
            \begin{adjustbox}{width=.84\paperwidth, height=.5\paperheight, center, margin= 0cm 0cm 0cm .1cm}
                \includesvg{diagrammaC4InserimentoLibro.svg} 
            \end{adjustbox}
            \captionof{figure}{Diagramma di sequenza C4} 
        \end{figure}
        \newpage

        \subsubsection{\textbf{C5: Modifica di un libro}}
        In questo diagramma, per migliorarne la leggibilità, abbiamo omesso il flusso alternativo di annullamento dell’operazione tramite il pulsante “Annulla”. Qui per il flusso alternativo di modifica errata o incompleta abbiamo usato la sintassi del loop-break.
        \begin{figure}[h!]
            \begin{adjustbox}{width=.84\paperwidth, height=.5\paperheight, center, margin= 0cm 0cm 0cm .1cm}
                \includesvg{diagrammaC5ModificaLibro.svg} 
            \end{adjustbox}
            \captionof{figure}{Diagramma di sequenza C5} 
        \end{figure}
        \newpage

        \subsubsection{\textbf{C6: Rimozione di un libro}}
        In questo diagramma, per il flusso alternativo di annullamento dell’operazione tramite il pulsante “No” abbiamo usato la sintassi dell’alt.
        \begin{figure}[h!]
            \begin{adjustbox}{width=.84\paperwidth, height=.5\paperheight, center, margin= 0cm 0cm 0cm .1cm}
                \includesvg{diagrammaC6RimuoviLibro.svg} 
            \end{adjustbox}
            \captionof{figure}{Diagramma di sequenza C6} 
        \end{figure}
        \newpage

        \subsubsection{\textbf{C7: Inserimento di un utente}}
        In questo diagramma, per migliorarne la leggibilità, abbiamo omesso il flusso alternativo di annullamento dell’operazione tramite il pulsante “Annulla”. Per il flusso alternativo di inserimento errato o incompleto abbiamo usato la sintassi del loop-break. Per quanto riguarda la sintassi per la descrizione del flusso alternativo dell’univocità dell’utente, abbiamo usato un alt.
        \begin{figure}[h!]
            \begin{adjustbox}{width=.84\paperwidth, height=.5\paperheight, center, margin= 0cm 0cm 0cm .1cm}
                \includesvg{diagrammaC7InserisciUtente.svg} 
            \end{adjustbox}
            \captionof{figure}{Diagramma di sequenza C7} 
        \end{figure}
        \newpage

        \subsubsection{\textbf{C8: Modifica di un utente}}
        In questo diagramma, per migliorarne la leggibilità, abbiamo omesso il flusso alternativo di annullamento dell’operazione tramite il pulsante “Annulla”. Qui per il flusso alternativo di modifica errata o incompleta abbiamo usato la sintassi del loop-break.
        \begin{figure}[h!]
            \begin{adjustbox}{width=.84\paperwidth, height=.5\paperheight, center, margin= 0cm 0cm 0cm .1cm}
                \includesvg{diagrammaC8ModificaUtente.svg} 
            \end{adjustbox}
            \captionof{figure}{Diagramma di sequenza C8} 
        \end{figure}
        \newpage

        \subsubsection{\textbf{C9: Rimozione di un utente}}
        In questo diagramma, per il flusso alternativo di annullamento dell’operazione tramite il pulsante “No” abbiamo usato la sintassi dell’alt.
        \begin{figure}[h!]
            \begin{adjustbox}{width=.84\paperwidth, height=.5\paperheight, center, margin= 0cm 0cm 0cm .1cm}
                \includesvg{diagrammaC9RimuoviUtente.svg} 
            \end{adjustbox}
            \captionof{figure}{Diagramma di sequenza C9} 
        \end{figure}
        \newpage

        \subsubsection{\textbf{C10: Ricerca di un libro}}
        In questo diagramma abbiamo usato la sintassi del loop-break per simulare la ricerca mediante sottostringa. Infatti, ogni volta che viene aggiunto un carattere, la ricerca si aggiorna per dare tutte le corrispondenze possibili con quella sottostringa, l’operazione si fermerà quando l’amministratore cliccherà su un libro.
        \begin{figure}[h!]
            \begin{adjustbox}{width=.84\paperwidth, height=.5\paperheight, center, margin= 0cm 0cm 0cm .1cm}
                \includesvg{diagrammaC10RicercaLibro.svg} 
            \end{adjustbox}
            \captionof{figure}{Diagramma di sequenza C10} 
        \end{figure}
        \newpage

        \subsubsection{\textbf{C12: Ricerca di un utente}}
        In questo diagramma abbiamo usato la sintassi del loop-break per simulare la ricerca mediante sottostringa. Infatti, ogni volta che viene aggiunto un carattere, la ricerca si aggiorna per dare tutte le corrispondenze possibili con quella sottostringa, l’operazione si fermerà quando l’amministratore cliccherà su un utente.
        \begin{figure}[h!]
            \begin{adjustbox}{width=.84\paperwidth, height=.5\paperheight, center, margin= 0cm 0cm 0cm .1cm}
                \includesvg{diagrammaC12RicercaUtente.svg} 
            \end{adjustbox}
            \captionof{figure}{Diagramma di sequenza C12} 
        \end{figure}
        \newpage

        \subsubsection{\textbf{C13: Registrazione di un prestito}}
        In questo diagramma, per migliorarne la leggibilità, abbiamo omesso il flusso alternativo di annullamento dell’operazione tramite il pulsante “Annulla”. Per il flusso alternativo di inserimento errato o incompleto abbiamo usato la sintassi del loop-break. Per quanto riguarda la sintassi per la descrizione del flusso alternativo dell’esistenza dell’utente e dell’esistenza del libro, abbiamo usato un alt.
        \begin{figure}[h!]
            \begin{adjustbox}{width=.84\paperwidth, height=.5\paperheight, center, margin= 0cm 0cm 0cm .1cm}
                \includesvg{diagrammaC13RegistraPrestito.svg} 
            \end{adjustbox}
            \captionof{figure}{Diagramma di sequenza C13} 
        \end{figure}
        \newpage

        \subsubsection{\textbf{C14: Registrazione di una restituzione}}
        In questo diagramma, per il flusso alternativo di annullamento dell’operazione tramite il pulsante “No” abbiamo usato la sintassi dell’alt.
        \begin{figure}[h!]
            \begin{adjustbox}{width=.84\paperwidth, height=.5\paperheight, center, margin= 0cm 0cm 0cm .1cm}
                \includesvg{diagrammaC14RegistraRestituzione.svg} 
            \end{adjustbox}
            \captionof{figure}{Diagramma di sequenza C14} 
        \end{figure}
        \newpage

        \subsubsection{\textbf{C15: Estensione di un prestito}}
        In questo diagramma, per migliorarne la leggibilità, abbiamo omesso il flusso alternativo di annullamento dell’operazione tramite il pulsante “Annulla”. Qui per il flusso alternativo di modifica errata o incompleta abbiamo usato la sintassi del loop-break. 
        \begin{figure}[h!]
            \begin{adjustbox}{width=.84\paperwidth, height=.5\paperheight, center, margin= 0cm 0cm 0cm .1cm}
                \includesvg{diagrammaC15EstendiPrestito.svg} 
            \end{adjustbox}
            \captionof{figure}{Diagramma di sequenza C15} 
        \end{figure}
    \newpage
    
    \subsection{Principi di buona progettazione}
        \subsubsection{Livelli di Coesione}
            \begin{adjustbox}{width=.8\paperwidth, height=.75\paperheight, center, margin= 0cm 0cm 0cm .1cm}
                \includegraphics{coesione.png}    
            \end{adjustbox}
            \captionof{figure}{Tabella dei livelli di coesione dei moduli} 
        \newpage

        \subsubsection{Livelli di Accoppiamento}
            \begin{adjustbox}{width=.8\paperwidth, height=.8\paperheight, center, margin= 0cm 0cm 0cm .1cm}
                \includegraphics{accoppiamento.png}    
            \end{adjustbox}
            \captionof{figure}{Tabella dei livelli di accoppiamento tra i moduli} 
        \newpage

        \subsubsection{Strategie per la riduzione dell'Accoppiamento}
            Per ridurre l’accoppiamento la classe Utente non è stata implementata astratta; infatti, se avessimo voluto un codice più manutenibile avremmo dovuto rendere Utente astratta e creare una classe Studente che estendeva Utente. In questo modo in futuro sarebbe stato possibile avere altre tipologie di utenti (come Professore) all’interno della biblioteca ma avremmo avuto un accoppiamento molto più alto dovuto all’estensione di Utente.\vspace{.2cm}
            \\Per ridurre l’accoppiamento la classe Utente non è stata implementata astratta; infatti, se avessimo voluto un codice più manutenibile avremmo dovuto rendere Utente astratta e creare una classe Studente che estendeva Utente. In questo modo in futuro sarebbe stato possibile avere altre tipologie di utenti (come Professore) all’interno della biblioteca ma avremmo avuto un accoppiamento molto più alto dovuto all’estensione di Utente. 
\end{document}
