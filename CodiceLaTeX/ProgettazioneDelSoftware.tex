\documentclass[12pt, a4paper]{article}

\usepackage[top=2.3cm, bottom=2cm, left=2cm, right=2cm, headheight=15pt]{geometry}
\usepackage{graphicx}
\usepackage{svg}
\usepackage[italian]{babel}
\usepackage{lastpage}
\usepackage{adjustbox}
\usepackage{enumitem}
\usepackage[most]{tcolorbox}
%\usepackage{xcolor}
\usepackage[colorlinks=true]{hyperref}
\usepackage{fancyhdr}
\pagestyle{fancy}
\usepackage{caption}
\renewcommand{\contentsname}{Indice}
\renewcommand{\sectionmark}[1]{\markboth{#1}{#1}}


\begin{document}
    \begin{titlepage}
        \fancyhf{}
		\centering
		
		{\Large \textsc{Ingegneria del Software - A.A. 2025/2026}}\\[.4cm]
        {\Large Applicazione per la gestione di una Biblioteca universitaria}\\[4cm]
		
		{\huge \textbf{Progettazione del Software}}\\[0.5cm]
		
		\large { 
			\textit{
            Luisa Genovese\\
            Erica Brancaccio\\
            Paolo Alfé\\
            Francesco Altieri}
		}
		
		\vspace{14cm}
		
		{\large \today}
		
		
	\end{titlepage}

    \fancyhf{}
    \fancyhead[R]{pagina \thepage\ di \pageref*{LastPage}}
    \pagestyle{fancy}
    \hypersetup{linkcolor=black}
    \begingroup
        \let\bfseries\normalfont
        \tableofcontents
    \endgroup
    \newpage

    \fancyhead[L]{\nouppercase{\leftmark}}
    \section{Progettazione del Software}
    \subsection{Decomposizione in moduli}
        \subsubsection{Descrizione delle classi individuate}
            Il diagramma delle classi rappresenta il sistema della biblioteca, con le funzionalità di gestione (ovvero di aggiunta, modifica e rimozione dall'archivio) dei libri, degli utenti e dei prestiti, nonché quella di salvataggio/lettura di dati.
            \begin{itemize}[label=$\bullet$]
                \setlength\itemsep{0.1em}
                \item \textbf{Biblioteca:} classe che modella una biblioteca. Ha tre attributi: “libri”, “utenti” e “prestiti”. I metodi servono per leggere e salvare la biblioteca su file;
                \item \textbf{Maschera:} classe che modella una maschera per la verifica dei campi di un oggetto. Ha due attributi: “lunghezza” e “maschera”. Ha dei metodi per la gestione della maschera (getter e setter); 
                \item \textbf{Prestiti:} classe che modella un insieme di oggetti Prestit. Ha dei metodi per la gestione dei prestiti (getter)., ha un attributo “prestiti” e un metodo di filtraggio per prendere solo i prestiti che corrispondono al filtro applicato. Attraverso l’interfaccia che implementa, possiede tre metodi: aggiungi, modifica e rimuovi; 
                \item \textbf{Utenti:} classe che modella un insieme di oggetti Utente. Ha due attributi: “chiaviISBN” e “libri”. Ha dei metodi per la gestione degli utenti (getter) e possiede un metodo per la ricerca all’ interno della collezione. Attraverso l’interfaccia Mappabile che implementa, possiede un metodo per la ricerca rapida attraverso chiave e di ottieni valore associato alla chiave cercata. Attraverso l’interfaccia Archiviabile possiede i metodi di aggiunta, modifica e rimozione;
                \item \textbf{Libri:} classe che modella un insieme di oggetti Libro. Ha due attributi: “chiaviMatricole” e “utenti” e possiede un metodo per la ricerca dei libri attraverso una stringa. Attraverso l’interfaccia Mappabile che implementa, possiede un metodo per la ricerca rapida attraverso chiave e di ottieni valore associato alla chiave cercata. Attraverso l’interfaccia Archiviabile possiede i metodi di aggiunta, modifica e rimozione;
                \item \textbf{Archiviabile:} è un'interfaccia parametrizzata implementata da Prestiti, Utenti e Libri. Ha tre metodi che servono per l’aggiunta, la modifica e la rimozione del parametro passato; 
                \item \textbf{Mappabile:} è un’interfaccia parametrizzata implementata da Utenti e da Libri. Ha un metodo di ricerca immediata attraverso chiave e un metodo usato per ottenere il valore associato alla chiave cercata;
                \item \textbf{Filtro:} è una classe enumerativa che possiede tre stati: TUTTI, ATTIVI e CONCLUSI. Questa classe serve per eseguire il filtraggio dei prestiti; 
                \item \textbf{Prestito:} classe che modella un prestito. Ha cinque attributi: “matricolaUtente”, “codiceISBNLibro”, “dataInizio”, “dataScadenza” e “dataRestituzione”. Ha vari metodi per la gestione dell’utente (getter e setter) e infine ha un metodo per verificare la correttezza del formato dei campi; 
                \item \textbf{Utente:} classe che modella un utente. Ha tre attributi: “matricolaUtente”, “email” e “prestitiAttivi”. Ha vari metodi per la gestione dell’utente (getter e setter). Ha due metodi per gestire i prestiti a lui associati ovvero aggiunta prestito, rimozione prestito. Ha infine dei metodi per verificare la correttezza del formato dei campi. Si tratta di una specializzazione della classe Persona; 
                \item \textbf{Libro:} classe che modella un libro. Ha cinque attributi: “titolo”, “autori”, “annoPubblicazione”, “codiceISBNLibro” e “copieDisponibili”. Ha vari metodi per la gestione del libro (getter e setter) e ha due metodi per gestire gli autori a lui associati, ovvero aggiunta e rimozione. Ha infine un metodo per verificare la correttezza del formato dei campi;
                \item \textbf{Matricola:} classe che modella una matricola di un utente. Ha un attributo “matricola”, un metodo per la gestione della matricola (getter) e uno (statico) per verificare la correttezza del proprio formato; 
                \item \textbf{ISBN:} classe che modella un codice ISBN. Ha un attributo “codiceISBN”. Ha un metodo per la gestione dell’ISBN (getter) e un metodo per verificare la correttezza del formato dell’ISBN; 
                \item \textbf{Autore:} classe che modella un autore di un libro. Così come Utente, specializza la classe Persona.
                \item \textbf{Persona:} classe astratta che modella una persona. Ha due attributi “nome” e “cognome” e ha vari metodi usati per la gestione di persona (getter e setter) e due per verificare la correttezza dei formati.
            \end{itemize} 
            All'interno del diagramma viene omessa l’implementazione dell’interfaccia Serializable al fine di aumentare la propria leggibilità. Suddetta interfaccia dovrà essere implementata dalle classi Biblioteca, Prestiti, Libri, Utenti, Prestito, Libro, Utente, Autore, Matricola e ISBN.\vspace{0.2cm}
            \\Per lo stesso motivo, all'interno del diagramma vengono omesse le classi Controller e le proprie relazioni con i modelli. 
        \newpage

        \subsubsection{Diagramma delle classi}
        \begin{figure}[h!]
            \begin{adjustbox}{width=.92\paperwidth, height=.75\paperheight, center, margin= 0cm .6cm 0cm .1cm}
                \includesvg{classdiagram.svg}    
            \end{adjustbox}
            \captionof{figure}{Diagramma delle Classi} 
        \end{figure}
    \newpage

    \subsection{Diagrammi di sequenza}
        \subsubsection{\textbf{C4: Inserimento di un libro}}
        \begin{figure}[h!]
            \begin{adjustbox}{width=.9\paperwidth, height=.66\paperheight, center, margin= 0cm 0cm 0cm 0cm}
                \includesvg{diagrammaC4InserimentoLibro.svg} 
            \end{adjustbox}
            \captionof{figure}{Diagramma di sequenza C4} 
        \end{figure}
        \newpage

        \subsubsection{\textbf{C5: Modifica di un libro}}
        \begin{figure}[h!]
            \begin{adjustbox}{width=.8\paperwidth, height=.75\paperheight, center, margin= 0cm .6cm 0cm .1cm}
                \includesvg{diagrammaC5ModificaLibro.svg} 
            \end{adjustbox}
            \captionof{figure}{Diagramma di sequenza C5} 
        \end{figure}
        \newpage

        \subsubsection{\textbf{C6: Rimozione di un libro}}
        \begin{figure}[h!]
            \begin{adjustbox}{width=.8\paperwidth, height=.75\paperheight, center, margin= 0cm .6cm 0cm .1cm}
                \includesvg{diagrammaC6RimuoviLibro.svg} 
            \end{adjustbox}
            \captionof{figure}{Diagramma di sequenza C6} 
        \end{figure}
        \newpage

        \subsubsection{\textbf{C7: Inserimento di un utente}}
        \begin{figure}[h!]
            \begin{adjustbox}{width=.8\paperwidth, height=.75\paperheight, center, margin= 0cm .6cm 0cm .1cm}
                \includesvg{diagrammaC7InserisciUtente.svg} 
            \end{adjustbox}
            \captionof{figure}{Diagramma di sequenza C7} 
        \end{figure}
        \newpage

        \subsubsection{\textbf{C8: Modifica di un utente}}
        \begin{figure}[h!]
            \begin{adjustbox}{width=.8\paperwidth, height=.75\paperheight, center, margin= 0cm .6cm 0cm .1cm}
                \includesvg{diagrammaC8ModificaUtente.svg} 
            \end{adjustbox}
            \captionof{figure}{Diagramma di sequenza C8} 
        \end{figure}
        \newpage

        \subsubsection{\textbf{C9: Rimozione di un utente}}
        \begin{figure}[h!]
            \begin{adjustbox}{width=.8\paperwidth, height=.75\paperheight, center, margin= 0cm .6cm 0cm .1cm}
                \includesvg{diagrammaC9RimuoviUtente.svg} 
            \end{adjustbox}
            \captionof{figure}{Diagramma di sequenza C9} 
        \end{figure}
        \newpage

        \subsubsection{\textbf{C10: Ricerca di un libro}}
        \begin{figure}[h!]
            \begin{adjustbox}{width=.8\paperwidth, height=.75\paperheight, center, margin= 0cm .6cm 0cm .1cm}
                \includesvg{diagrammaC10RicercaLibro.svg} 
            \end{adjustbox}
            \captionof{figure}{Diagramma di sequenza C10} 
        \end{figure}
        \newpage

        \subsubsection{\textbf{C12: Ricerca di un utente}}
        \begin{figure}[h!]
            \begin{adjustbox}{width=.8\paperwidth, height=.75\paperheight, center, margin= 0cm .6cm 0cm .1cm}
                \includesvg{diagrammaC12RicercaUtente.svg} 
            \end{adjustbox}
            \captionof{figure}{Diagramma di sequenza C12} 
        \end{figure}
        \newpage

        \subsubsection{\textbf{C13: Registrazione di un prestito}}
        \begin{figure}[h!]
            \begin{adjustbox}{width=.8\paperwidth, height=.75\paperheight, center, margin= 0cm .6cm 0cm .1cm}
                \includesvg{diagrammaC13RegistraPrestito.svg} 
            \end{adjustbox}
            \captionof{figure}{Diagramma di sequenza C13} 
        \end{figure}
        \newpage

        \subsubsection{\textbf{C14: Registrazione di una restituzione}}
        \begin{figure}[h!]
            \begin{adjustbox}{width=.8\paperwidth, height=.75\paperheight, center, margin= 0cm .6cm 0cm .1cm}
                \includesvg{diagrammaC14RegistraRestituzione.svg} 
            \end{adjustbox}
            \captionof{figure}{Diagramma di sequenza C14} 
        \end{figure}
        \newpage

        \subsubsection{\textbf{C15: Estensione di un prestito}}
        \begin{figure}[h!]
            \begin{adjustbox}{width=.8\paperwidth, height=.75\paperheight, center, margin= 0cm .6cm 0cm .1cm}
                \includesvg{diagrammaC15EstendiPrestito.svg} 
            \end{adjustbox}
            \captionof{figure}{Diagramma di sequenza C15} 
        \end{figure}
    \newpage
    
    \subsection{Principi di buona progettazione}
        \subsubsection{Livelli di Coesione}
            \begin{adjustbox}{width=.8\paperwidth, height=.76\paperheight, center, margin= 0cm .2cm 0cm .3cm}
                \includegraphics{coesione.png}    
            \end{adjustbox}
            \captionof{figure}{Tabella dei livelli di coesione dei moduli} 
        \newpage

        \subsubsection{Livelli di Accoppiamento}
            \begin{adjustbox}{width=.8\paperwidth, height=.79\paperheight, center, margin= 0cm .2cm 0cm .2cm}
                \includegraphics{accoppiamento.png}    
            \end{adjustbox}
            \captionof{figure}{Tabella dei livelli di accoppiamento tra i moduli} 
        \newpage

        \subsubsection{Strategie per la riduzione dell'Accoppiamento}
            ...
\end{document}
